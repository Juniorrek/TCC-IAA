%%%%%%%%%%%%%%%%%%%%%%%%%%%%%%%%%%%%%%%%%%%%%%%%%%%%%%%
% Arquivo para entrada de dados para a parte pré textual
%%%%%%%%%%%%%%%%%%%%%%%%%%%%%%%%%%%%%%%%%%%%%%%%%%%%%%%
% 
% Basta digitar as informações indicidas, no formato 
% apresentado.
%
%%%%%%%
% Os dados solicitados são, na ordem:
%
% tipo do trabalho
% componentes do trabalho 
% título do trabalho
% nome do autor
% local 
% data (ano com 4 dígitos)
% orientador(a)
% coorientador(a)(as)(es)
% arquivo com dados bibliográficos
% instituição
% setor
% programa de pós gradução
% curso
% preambulo
% data defesa
% CDU
% errata
% assinaturas - termo de aprovação
% resumos & palavras chave
% agradecimentos
% dedicatoria
% epígrafe


% Informações de dados para CAPA e FOLHA DE ROSTO
%----------------------------------------------------------------------------- 
\tipotrabalho{Trabalho Acadêmico}
%    {Relatório Técnico}
%    {Dissertação}
%    {Tese}
%    {Monografia}

% Marcar Sim para as partes que irão compor o documento pdf
%----------------------------------------------------------------------------- 
 \providecommand{\terCapa}{Sim}
 \providecommand{\terFolhaRosto}{Sim}
 \providecommand{\terTermoAprovacao}{Sim}
 \providecommand{\terDedicatoria}{Nao}
 \providecommand{\terFichaCatalografica}{Nao}
 \providecommand{\terEpigrafe}{Sim}
 \providecommand{\terAgradecimentos}{Sim}
 \providecommand{\terErrata}{Nao}
 \providecommand{\terListaFiguras}{Sim}
 \providecommand{\terListaQuadros}{Nao}
 \providecommand{\terListaTabelas}{Nao}
 \providecommand{\terSiglasAbrev}{Nao} 
 \providecommand{\terSimbolos}{Nao}
 \providecommand{\terResumos}{Sim}
 \providecommand{\terSumario}{Sim}
 \providecommand{\terAnexo}{Nao}
 \providecommand{\terApendice}{Sim}
 \providecommand{\terIndiceR}{Nao}
%----------------------------------------------------------------------------- 

\titulo{Memorial de Projetos: Processamento de Linguagem Natural como Ponte entre a Inteligência Artificial a Linguística}
\autor{David Reksidler Júnior}
\local{Curitiba}
\data{2025} %Apenas ano 4 dígitos

% Orientador ou Orientadora
\orientador{}
%Prof Emílio Eiji Kavamura, MSc}
\orientadora{
Prof\textordfeminine~Dr\textordfeminine~Rafaela Mantovani Fontana}
% Pode haver apenas uma orientadora ou um orientador
% Se houver os dois prevalece o feminino.

% Em termos de coorientação, podem haver até quatro neste modelo
% Sendo 2 mulhere e 2 homens.
% Coorientador ou Coorientadora
\coorientador{}%Prof Morgan Freeman, DSc}
\coorientadora{}

% Segundo Coorientador ou Segunda Coorientadora
\scoorientador{}
%Prof Jack Nicholson, DEng}
\scoorientadora{}
%Prof\textordfeminine~Ingrid Bergman, DEng}
% ----------------------------------------------------------
\addbibresource{referencias.bib}

% ----------------------------------------------------------
\instituicao{%
Universidade Federal do Paraná}

\def \ImprimirSetor{}%
%Setor de Tecnologia}

\def \ImprimirProgramaPos{}%Programa de Pós Graduação em Engenharia de Construção Civil}

\def \ImprimirCurso{}%
%Curso de Engenharia Civil}

\preambulo{
Memorial de Projetos apresentado ao curso de Especialização em Inteligência Artificial Aplicada, Setor de Educação Profissional e Tecnológica, Universidade Federal do Paraná, como requisito parcial à obtenção do título de Especialista em Inteligência Artificial Aplicada}
%do grau de Bacharel em Expressão Gráfica no curso de Expressão Gráfica, Setor de Exatas da Universidade Federal do Paraná}

%----------------------------------------------------------------------------- 

\newcommand{\imprimirCurso}{}
%Programa de P\'os Gradua\c{c}\~ao em Engenharia da Constru\c{c}\~ao Civil}

\newcommand{\imprimirDataDefesa}{
09 de Dezembro de 2018}

\newcommand{\imprimircdu}{
02:141:005.7}

% ----------------------------------------------------------
\newcommand{\imprimirerrata}{
Elemento opcional da \cites[4.2.1.2]{NBR14724:2011}. Exemplo:

\vspace{\onelineskip}

FERRIGNO, C. R. A. \textbf{Tratamento de neoplasias ósseas apendiculares com
reimplantação de enxerto ósseo autólogo autoclavado associado ao plasma
rico em plaquetas}: estudo crítico na cirurgia de preservação de membro em
cães. 2011. 128 f. Tese (Livre-Docência) - Faculdade de Medicina Veterinária e
Zootecnia, Universidade de São Paulo, São Paulo, 2011.

\begin{table}[htb]
\center
\footnotesize
\begin{tabular}{|p{1.4cm}|p{1cm}|p{3cm}|p{3cm}|}
  \hline
   \textbf{Folha} & \textbf{Linha}  & \textbf{Onde se lê}  & \textbf{Leia-se}  \\
    \hline
    1 & 10 & auto-conclavo & autoconclavo\\
   \hline
\end{tabular}
\end{table}}

% Comandos de dados - Data da apresentação
\providecommand{\imprimirdataapresentacaoRotulo}{}
\providecommand{\imprimirdataapresentacao}{}
\newcommand{\dataapresentacao}[2][\dataapresentacaoname]{\renewcommand{\dataapresentacao}{#2}}

% Comandos de dados - Nome do Curso
\providecommand{\imprimirnomedocursoRotulo}{}
\providecommand{\imprimirnomedocurso}{}
\newcommand{\nomedocurso}[2][\nomedocursoname]
  {\renewcommand{\imprimirnomedocursoRotulo}{#1}
\renewcommand{\imprimirnomedocurso}{#2}}


% ----------------------------------------------------------
\newcommand{\AssinaAprovacao}{

\assinatura{%\textbf
   {Professora} \\ UFPR}
   \assinatura{%\textbf
   {Professora} \\ ENSEADE}
   \assinatura{%\textbf
   {Professora} \\ TIT}
   %\assinatura{%\textbf{Professor} \\ Convidado 4}
      
   \begin{center}
    \vspace*{0.5cm}
    %{\large\imprimirlocal}
    %\par
    %{\large\imprimirdata}
    \imprimirlocal, \imprimirDataDefesa.
    \vspace*{1cm}
  \end{center}
  }
  
% ----------------------------------------------------------
%\newcommand{\Errata}{%\color{blue}
%Elemento opcional da \textcite[4.2.1.2]{NBR14724:2011}. Exemplo:
%}

% ----------------------------------------------------------
\newcommand{\EpigrafeTexto}{%\color{blue}
\textit{``Even his most cogent moments were tinged with delusion. On the other hand,\\ his most delirious dreams touched on deep realities.''\\
		(Mad Merlin - J. Robert King)}
}

% ----------------------------------------------------------
\newcommand{\ResumoTexto}{%\color{blue}
A inteligência humana, com sua capacidade de aprender e resolver problemas, inspirou o surgimento da IA, área dedicada a reproduzir computacionalmente aspectos do raciocínio e da aprendizagem. Este parecer técnico apresenta uma breve visão da evolução da Inteligência Artificial, desde sistemas baseados em regras fixas até abordagens modernas como Aprendizado de Máquina e Aprendizado Profundo. Entre suas principais subáreas, destaca-se o Processamento de Linguagem Natural, responsável por ensinar máquinas a compreender e gerar linguagem humana. O estudo também discute a relação entre o Processamento de Linguagem Natural e a Linguística, abordando desafios como ambiguidade e contexto, além de analisar o impacto dos Modelos de Linguagem de Grande Escala, como o ChatGPT. Por fim, o trabalho reflete sobre o caráter interdisciplinar da IA, apresentando os trabalhos realizados durante a realização do curso e como os conhecimentos adquiridos ao longo das disciplinas contribuíram para a elaboração deste parecer técnico.
} 

\newcommand{\PalavraschaveTexto}{%\color{blue}
inteligência artificial; processamento de linguagem natural; linguística.}

% ----------------------------------------------------------
\newcommand{\AbstractTexto}{%\color{blue}
Human intelligence, with its ability to learn and solve problems, inspired the emergence of AI, an area dedicated to computationally reproducing aspects of reasoning and learning. This technical report presents a brief overview of the evolution of Artificial Intelligence, from fixed rule-based systems to modern approaches like Machine Learning and Deep Learning. Among its main subareas, Natural Language Processing stands out, responsible for teaching machines to understand and generate human language. The study also discusses the relationship between Natural Language Processing and Linguistics, addressing challenges such as ambiguity and context, in addition to analyzing the impact of Large Scale Language Models, such as ChatGPT. Finally, the work reflects on the interdisciplinary nature of AI, presenting the work carried out during the course and how the knowledge acquired throughout the subjects contributed to the preparation of this technical report.
}
% ---
\newcommand{\KeywordsTexto}{%\color{blue}
artificial intelligence; natural language processing; linguistics.
}

% ----------------------------------------------------------
\newcommand{\Resume}
{%\color{blue}
%Il s'agit d'un résumé en français.
} 
% ---
\newcommand{\Motscles}
{%\color{blue}
 %latex. abntex. publication de textes.
}

% ----------------------------------------------------------
\newcommand{\Resumen}
{%\color{blue}
%Este es el resumen en español.
}
% ---
\newcommand{\Palabrasclave}
{%\color{blue}
%latex. abntex. publicación de textos.
}

% ----------------------------------------------------------
\newcommand{\AgradecimentosTexto}{%\color{blue}
A realização deste trabalho contou com o apoio e a contribuição de muitas pessoas, às quais registro minha mais sincera gratidão.

Agradeço primeiramente à minha esposa, Renata, pelo companheirismo, paciência e suporte constantes durante toda minha jornada acadêmica. Seu incentivo foi essencial para que eu mantivesse o foco e concluísse esta etapa com dedicação e serenidade.

Agradeço também à minha família, em especial aos meus pais, David e Patrícia, e aos meus irmãos, Kelvyn e Gabriel, pelo apoio, palavras de incentivo e confiança que sempre depositaram em mim.

Registro ainda meu agradecimento à Prof.ª Dra. Rafaela Mantovani Fontana, orientadora deste trabalho, pelo acompanhamento e orientações durante o desenvolvimento do mesmo.

Por fim, deixo meu reconhecimento a todos que, de alguma forma, contribuíram para a realização deste trabalho e para minha formação ao longo do curso.
}

% ----------------------------------------------------------
\newcommand{\DedicatoriaTexto}{%\color{blue}
\textit{ Este trabalho é dedicado às crianças adultas que,\\
   quando pequenas, sonharam em se tornar cientistas.}
	}

