% Pacotes básicos 
% ----------------------------------------------------------
%\usepackage{lmodern}			% Usa a fonte Latin Modern			
\usepackage[T1]{fontenc}		% Selecao de codigos de fonte.
\usepackage[utf8]{inputenc}		% Codificacao do documento (conversão automática dos acentos)
\usepackage{lastpage}			% Usado pela Ficha catalográfica
\usepackage{indentfirst}		% Indenta o primeiro parágrafo de cada seção.
\usepackage{microtype} 			% para melhorias de justificação
\usepackage{ifthen}		    	% para montar condicionais
\usepackage[brazil]{babel}		% para utilizar termos em portugues
\usepackage[final]{pdfpages}    % para incluir páginas de arquivos pdf
\usepackage{amsmath}			% simbolos matematicos

% Pacotes de citações BibLaTeX
% ----------------------------------------------------------
\usepackage[style=abnt,
	backref=true,
	backend=biber,
	citecounter=true,
	backrefstyle=three, 
 	repeatfields=true,
	url=true,
	maxbibnames=99,
    mincitenames=1,
    maxcitenames=2,
    backref=true,
    hyperref=true,
    giveninits=true,
    uniquename=false,
    uniquelist=false]{biblatex}

% Norma NBR10.520/2023 - Sobrenomes em minúsculas
\renewcommand*{\mkbibnamefamily}[1]{#1}%

% Espaçamento entre os itens nas referências (espço de uma linha simples)
% ----------------------------------------------------------
\setlength\bibitemsep{\baselineskip}

% Texto padrão para as referências
% ----------------------------------------------------------
\DefineBibliographyStrings{brazil}{%
	 backrefpage  = {Citado \arabic{citecounter} vez na página},		% originally "cited on page"
	 backrefpages = {Citado \arabic{citecounter} vezes nas páginas},	% originally "cited on pages"
	 urlfrom      = {Dispon\'ivel em},
}

% Ajusta indentação de Referencias no ToC
% ----------------------------------------------------------
\defbibheading{bay}[\bibname]{%
  \chapter*{#1}%
  \markboth{#1}{#1}%
  \addcontentsline{toc}{chapter}
  %{\protect\numberline{}\bibname}
  {\bibname}
}

% Formatando o avançao dos títulos no sumário 
% ----------------------------------------------------------
\makeatletter
	\pretocmd{\chapter}{\addtocontents{toc}{\protect\addvspace{-12\p@}}}{}{}
	\pretocmd{\section}{\addtocontents{toc}{\protect\addvspace{-3\p@}}}{}{}
\makeatother

% https://groups.google.com/g/abntex2/c/ZYwE4t9uTFM
\makeatletter
\let\oldcontentsline\contentsline
\def\contentsline#1#2{%
	\expandafter\ifx\csname l@#1\endcsname\l@section
	\expandafter\@firstoftwo
	\else
	\expandafter\@secondoftwo
	\fi
	{%
		\oldcontentsline{#1}{\MakeTextUppercase{#2}}%
	}{%
		\oldcontentsline{#1}{#2}%
	}%
}
\makeatother

% Para retirar os símbolos <> da URL  
% ----------------------------------------------------------
\DeclareFieldFormat{illustrated}{\addspace #1\isdot}%
%\DeclareFieldFormat{url}{\bibstring{urlform}\addcolon\addspace<\url{#1}>}%
%\DeclareFieldFormat{url}{\bibstring{urlfrom}\addcolon\addspace<\url{#1}>}%
\DeclareFieldFormat{url}{\bibstring{urlfrom}\addcolon \space\addspace{#1}} 
% remove <> em urls de acordo com abnt-6023:2018	

% Ajustar o espaço para a formatação da data
% ----------------------------------------------------------
\DeclareFieldFormat{urldate}{\bibstring{urlseen}\addcolon\addspace #1}%
\DeclareFieldFormat*{note}{\addspace #1}%

% Para ajustar o tamanho da fonte do número da primeira página do capítulo
% comando utilizado na parte textual 
% ----------------------------------------------------------
\makepagestyle{chapfirst}% Just for the first page of a chapter
\makeoddhead{chapfirst}{}{}{\footnotesize{\thepage}}

%%criar um novo estilo de cabeçalhos e rodapés
\makepagestyle{simplestextual}
  %%cabeçalhos
  \makeevenhead{simplestextual} %%pagina par
     {}{}{\footnotesize \thepage}
     
  \makeoddhead{simplestextual} %%pagina ímpar ou com oneside
     {}{}{\footnotesize \thepage}
  %\makeheadrule{simplestextual}{\textwidth}{\normalrulethickness} %linha
  %% rodapé
  \makeevenfoot{simplestextual}
     {}{}{} %%pagina par
      
  \makeoddfoot{simplestextual} %%pagina ímpar ou com oneside
     {}{}{}
     
% Define a formatação dos capítulos póstextuais numerados
% ----------------------------------------------------------
%\newcommand{\refap}[1]{\hyperref[#1]{Apêndice~\ref{#1}}} 	% Referência apÊndices



 %%%%%%%%%%% NBR 10520/23

 % Norma NBR10.520/2023 - Sobrenomes em minúsculas
\renewcommand*{\mkbibnamefamily}[1]{#1}%

%fullcite com todos autores e não como cite
\makeatletter
\newcommand{\tempmaxup}[1]{\def\blx@maxcitenames{99}#1}
\makeatother

\DeclareCiteCommand{\fullcite}[\tempmaxup]
{\usebibmacro{prenote}}
{\usedriver
	{}
	{\thefield{entrytype}}}
{\multicitedelim}
{\usebibmacro{postnote}}

\usepackage{graphicx}
\usepackage{enumitem}
\usepackage{float}
\usepackage{changepage}
\usepackage{tabularx}
\usepackage{listings}
\usepackage[dvipsnames,table]{xcolor}

\usepackage{tikz}
\usetikzlibrary{shapes.arrows, shapes.geometric, arrows, fit}
\newcommand{\FancyUpArrow}{\begin{tikzpicture}[baseline=-0.3em]
\node[single arrow,draw,rotate=90,single arrow head extend=0.2em,inner
ysep=0.2em,transform shape,line width=0.05em,top color=green,bottom color=green!50!black] (X){};
\end{tikzpicture}}
\newcommand{\FancyUpDown}{\begin{tikzpicture}[baseline=-0.3em]
\node[single arrow,draw,rotate=270,single arrow head extend=0.2em,inner
ysep=0.2em,transform shape,line width=0.05em,top color=red,bottom color=red!50!black] (X){};
\end{tikzpicture}}

\usepackage{adjustbox}
\usepackage{array}
\usepackage{longtable}
\usepackage{makecell} 

\usepackage[most]{tcolorbox}
\tcbset{
    myoutputstyle/.style={
        colback=blue!10,
        colframe=blue!60!black,
        boxrule=0.5pt,
        left=3mm,
        right=3mm,
        top=2mm,
        bottom=2mm,
        width=0.9\linewidth,
        enlarge left by=0.05\linewidth,
    }
}
