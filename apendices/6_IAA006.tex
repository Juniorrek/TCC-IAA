\label{ap:ap06}
\chapter{Arquitetura de dados}
\section*{\textbf{A - ENUNCIADO}}


\subsection{Construção de Características: Identificador automático de idioma}

\textcolor{black}{O problema consiste em criar um modelo de reconhecimento de padrões que dado um texto de entrada, o
programa consegue classificar o texto e indicar a língua em que o texto foi escrito.}



\textcolor{black}{Parta do exemplo (notebook produzido no Colab) que foi disponibilidade e crie as funções para calcular
as diferentes características para o problema da identificação da língua do texto de entrada.}



\textcolor{black}{Nessa atividade é para {\textquotedbl}construir características{\textquotedbl}.}



\textcolor{black}{Meta: a acurácia deverá ser maior ou igual a 70\%.}



\textcolor{black}{Essa tarefa pode ser feita no Colab (Google) ou no Jupiter, em que deverá exportar o notebook e
imprimir o notebook para o formato PDF. Envie no UFPR Virtual os dois arquivos.}



\subsection{Melhore uma base de dados ruim}



\textcolor{black}{Escolha uma base de dados pública para problemas de classificação, disponível ou com origem na UCI
Machine Learning.}



\textcolor{black}{Use o mínimo de intervenção para rodar a SVM e obtenha a matriz de confusão dessa base.}



\textcolor{black}{O trabalho começa aqui, escolha as diferentes tarefas discutidas ao longo da disciplina, para melhorar
essa base de dados, até que consiga efetivamente melhorar o resultado.}



\textcolor{black}{Considerando a acurácia para bases de dados balanceadas ou quase balanceadas, se o percentual da
acurácia original estiver em até 85\%, a meta será obter 5\%. Para bases com mais de 90\% de acurácia, a meta será
obter a melhora em pelo menos 2 pontos percentuais (92\% ou mais).}



\textcolor{black}{Nessa atividade deverá ser entregue o script aplicado (o notebook e o PDF correspondente).}



%%%%%%%%%%%%%%%%%%%%%%%%%%%%%%%%%%%%%%%%%%%%%%%%%%%%%%%%%%%%%%%%%%%%%%%%%%%%%%%%%%%%%%%%%%%%%
\section*{\textbf{B - RESOLUÇÃO}}
\lipsum[30]