\label{ap:ap12}
\chapter{Gestão de Projetos de IA}
\section*{\textbf{A - ENUNCIADO}}

\subsection*{\textbf{1 Objetivo}}

\textcolor{black}{Individualmente, ler e resumir – seguindo o }\textit{\textcolor{black}{template }}\textcolor{black}{
fornecido – }\textbf{\textcolor{black}{um }}\textcolor{black}{ dos artigos abaixo:}



\foreignlanguage{english}{\textcolor{black}{AHMAD, L.; ABDELRAZEK, M.; ARORA, C.; BANO, M; GRUNDY, J. Requirements
practices and gaps when engineering human-centered Artificial Intelligence systems. Applied Soft Computing. 143. 2023.
DOI }}\url{https://doi.org/10.1016/j.asoc.2023.110421}

\foreignlanguage{english}{\textcolor{black}{NAZIR, R.; BUCAIONI, A.; PELLICCIONE, P.; Architecting ML-enabled systems:
Challenges, best practices, and design decisions. The Journal of Systems \& Software. 207. 2024. DOI
}}\url{https://doi.org/10.1016/j.jss.2023.111860}\foreignlanguage{english}{\textcolor{black}{ }}

\foreignlanguage{english}{\textcolor{black}{SERBAN, A.; BLOM, K.; HOOS, H.; VISSER, J. Software engineering practices
for machine learning – Adoption, effects, and team assessment. The Journal of Systems \& Software. 209. 2024. DOI
}}\url{https://doi.org/10.1016/j.jss.2023.111907}\foreignlanguage{english}{\textcolor{black}{ }}

\foreignlanguage{english}{\textcolor{black}{STEIDL, M.; FELDERER, M.; RAMLER, R. The pipeline for continuous development
of artificial intelligence models – Current state of research and practice. The Journal of Systems \& Software. 199.
2023. DOI }}\url{https://doi.org/10.1016/j.jss.2023.111615}\foreignlanguage{english}{\textcolor{black}{ }}

\foreignlanguage{english}{\textcolor{black}{XIN, D.; WU, E. Y.; LEE, D. J.; SALEHI, N.; PARAMESWARAN, A. Whither AutoML?
Understanding the Role of Automation in Machine Learning Workflows. In CHI Conference on Human Factors in Computing
Systems (CHI'21), Maio 8-13, 2021, Yokohama, Japão. }}\textcolor{black}{DOI
}\url{https://doi.org/10.1145/3411764.3445306}\textcolor{black}{ }



\subsection*{\textbf{2 Orientações adicionais}}



Escolha o artigo que for mais interessante para você. Utilize tradutores e o Chat GPT para entender o conteúdo dos
artigos – caso precise, mas escreva o resumo em língua portuguesa e nas suas palavras. 



Não esqueça de preencher, no trabalho, os campos relativos ao seu nome e ao artigo escolhido.



No \textit{template}, você deverá responder às seguintes questões:

\begin{itemize}
\item Qual o objetivo do estudo descrito pelo artigo?
\item Qual o problema/oportunidade/situação que levou a necessidade de realização deste estudo?
\item Qual a metodologia que os autores usaram para obter e analisar as informações do estudo?
\item Quais os principais resultados obtidos pelo estudo?
\end{itemize}


Responda cada questão utilizando o espaço fornecido no \textit{template}, sem alteração do tamanho da fonte (Times New
Roman, 10), nem alteração do espaçamento entre linhas (1.0).



Não altere as questões do template.



Utilize o editor de textos de sua preferência para preencher as respostas, mas entregue o trabalho em PDF.


%%%%%%%%%%%%%%%%%%%%%%%%%%%%%%%%%%%%%%%%%%%%%%%%%%%%%%%%%%%%%%%%%%%%%%%%%%%%%%%%%%%%%%%%%%%%%
\section*{\textbf{B - RESOLUÇÃO}}
\lipsum[30]