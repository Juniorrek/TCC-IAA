\chapter{Histórias de Usuário} \label{cha:historiasUsuário}

Neste capítulo serão apresentadas as histórias de usuários definidas para o desenvolvimento deste software. As histórias são descritas de forma detalhada e contextualizada as necessidades, objetivos e requisitos específicos dos usuários em relação ao sistema. Cada história de usuário é uma narrativa curta que descreve uma funcionalidade. As histórias de usuário são uma ferramenta importante para o desenvolvimento ágil de software, pois permitem uma compreensão clara das expectativas dos usuários e fornecem direcionamento para o desenvolvimento incremental e iterativo do aplicativo.

\section{Buscar produto}%%%%%%%%%%%%%%%%%%%%%%%%%%%%%%%%%%%%%

\newcounter{numhistoria}
\setcounter{numhistoria}{1}
\newcommand{\nhist}{%
  \padzeroes[2]{\decimal{numhistoria}}%
  \stepcounter{numhistoria}%
}

%\begin{quadro}[hbt!]
%\centering
\begin{tabular}{|ll|}
\hline
\multicolumn{2}{|c|}{\textbf{UC\nhist - \currentname}}    \\ \hline
\multicolumn{1}{|l|}{\textbf{Sendo}}     & um usuário \\ \hline
\multicolumn{1}{|l|}{\textbf{Quero}}     & buscar um produto \\ \hline
\multicolumn{1}{|l|}{\textbf{Para}}      & encontrar oque tenha melhor preço e localização \\ \hline
\multicolumn{1}{|l|}{\textbf{Protótipo}} & 
\begin{minipage}{0.48\textwidth} 
\begin{figure}[H]
\caption{\label{fig:label} TELA INICIAL}
\includegraphics[width=\textwidth]{fig/telas/t_inicial.jpg}
\footnotesize \centering
\par FONTE: O Autor (2024)
\end{figure}
\end{minipage} \\ \hline
\end{tabular}
%\end{quadro}



\subsection*{\textbf{CRITÉRIOS DE ACEITAÇÃO}}

\begin{enumerate}[leftmargin=2cm]
    \item Deve permitir inserir o nome do produto.
    \item Deve definir a localização do usuário por GPS.
    \item Deve permitir que o usuário logue no sistema.
    \item Deve permitir que o usuário recupere a senha.
    \item Deve permitir que o usuário se cadastre.
    \item Deve buscar os produtos por nome.
\end{enumerate}

\subsection*{\textbf{CRITÉRIOS DE ACEITAÇÃO - DETALHAMENTO}}

\begin{tabularx}{0.9\textwidth}{|l|X|}
\multicolumn{2}{@{}l}{\textbf{1. Deve permitir inserir o nome do produto.}} \\ \hline
\textbf{Dado que} & Usuário está na tela de pesquisa. \\ \hline
\textbf{Quanto} & Usuário clica no campo "Produto". \\ \hline
\textbf{Então} & Sistema permite a inserção do nome do produto. (R1)\\ \hline
\end{tabularx}

\begin{tabularx}{0.9\textwidth}{|l|X|}
\multicolumn{2}{@{}l}{\textbf{2. Deve definir a localização do usuário por GPS.}} \\ \hline
\textbf{Dado que} & Usuário está na tela inicial.\\ \hline
\textbf{Quanto} & Usuário realiza qualquer requisição. \\ \hline
\textbf{Então} & Sistema salva a localização do usuário através do ip. \\ \hline
\end{tabularx}

\begin{tabularx}{0.9\textwidth}{|l|X|}
\multicolumn{2}{@{}l}{\textbf{3. Deve permitir que o usuário logue no sistema.}} \\ \hline
\textbf{Dado que} & Usuário está na tela de pesquisa. \\ \hline
\textbf{Quanto} & Usuário clica no botão "Login". \\ \hline
\textbf{Então} & Sistema redireciona para a tela de login. \\ \hline
\end{tabularx}

\begin{tabularx}{0.9\textwidth}{|l|X|}
\multicolumn{2}{@{}l}{\textbf{4. Deve permitir que o usuário recupere a senha.}} \\ \hline
\textbf{Dado que} & Usuário está na tela de pesquisa. \\ \hline
\textbf{Quanto} & Usuário lica no botão "Esqueci minha senha". \\ \hline
\textbf{Então} & Sistema redireciona para a tela. \\ \hline
\end{tabularx}

\begin{tabularx}{0.9\textwidth}{|l|X|}
\multicolumn{2}{@{}l}{\textbf{5. Deve permitir que o usuário se cadastre.}} \\ \hline
\textbf{Dado que} & Usuário está na tela de pesquisa. \\ \hline
\textbf{Quanto} &  Usuário clica no botão "Criar conta". \\ \hline
\textbf{Então} & Sistema redireciona para a tela de cadastro. \\ \hline
\end{tabularx}

\begin{tabularx}{0.9\textwidth}{|l|X|}
\multicolumn{2}{@{}l}{\textbf{6. Deve buscar os produtos por nome.}} \\ \hline
\textbf{Dado que} & Usuário inseriu o nome de um produto. (R1) \\ \hline
\textbf{Quanto} & Usuário clica em "Pesquisar". \\ \hline
\textbf{Então} & Sistema redireciona para a tela com os produtos buscados. \\ \hline
\end{tabularx}

\subsection*{\textbf{REGRAS DE NEGÓCIO DA HISTÓRIA}}

\begin{itemize}
    \item[] R1 - Tamanho máximo do texto 256 caracteres.
\end{itemize}

\section{Visualizar produtos}%%%%%%%%%%%%%%%%%%%%%%%%%%%%%%%%%%%%%

\begin{tabular}{|ll|}
\hline
\multicolumn{2}{|c|}{\textbf{UC\nhist - \currentname}}    \\ \hline
\multicolumn{1}{|l|}{\textbf{Sendo}}     & um usuário \\ \hline
\multicolumn{1}{|l|}{\textbf{Quero}}     & visualizar os produtos buscados\\ \hline
\multicolumn{1}{|l|}{\textbf{Para}}      & selecionar oque tenha melhor preço e localização \\ \hline
\multicolumn{1}{|l|}{\textbf{Protótipo}} & 
\begin{minipage}{0.48\textwidth} 
\begin{figure}[H]
\caption{\label{fig:label} TELA VISUALIZAR PRODUTOS}
\includegraphics[width=\textwidth]{fig/telas/t_listarprod.jpg}
\footnotesize \centering
\par FONTE: O Autor (2024)
\end{figure}
\end{minipage}
 \\ \hline
\end{tabular}

\subsection*{\textbf{CRITÉRIOS DE ACEITAÇÃO}}

\begin{enumerate}[leftmargin=2cm]
    \item Deve listar os produtos buscados.
    \item Deve permitir fazer uma nova busca.
    \item Deve permitir selecionar um produto na lista para mostrar mais detalhes.
    \item Deve permitir visualizar os produtos por mapa.
    \item Deve permitir filtrar a busca.
\end{enumerate}

\subsection*{\textbf{CRITÉRIOS DE ACEITAÇÃO - DETALHAMENTO}}
%\textbf{Critério de contexto} (Válido como premissa para todos os critérios):

%\begin{tabular}{@{}l l }
% \textbf{Dado que} & blabla \\ 
% \textbf{E} & blabla
%\end{tabular}

\begin{tabularx}{0.9\textwidth}{|l|X|}
\multicolumn{2}{@{}l}{\textbf{1. Deve listar os produtos buscados.}} \\ \hline
\textbf{Dado que} & Usuário pesquisou um produto valido. (R1)  \\ \hline
\textbf{Quanto} & Sistema carrega a tela. \\ \hline
\textbf{Então} & Sistema apresenta a lista dos produtos encontrados.  \\ \hline
\end{tabularx}

\begin{tabularx}{0.9\textwidth}{|l|X|}
\multicolumn{2}{@{}l}{\textbf{2. Deve permitir fazer uma nova busca.}} \\ \hline
\textbf{Dado que} & Usuário inseriu o nome de um produto. (R2) \\ \hline
\textbf{Quanto} & Usuário clica em "Buscar". \\ \hline
\textbf{Então} & Sistema carrega nova lista. \\ \hline
\end{tabularx}

\begin{tabularx}{0.9\textwidth}{|l|X|}
\multicolumn{2}{@{}l}{\textbf{\makecell[l]{3. Deve permitir selecionar um produto na lista para mostrar mais \\detalhes.}}} \\ \\
\hline \textbf{Dado que} & Sistema carregou ao menos 1 produto. (R1) \\ \hline
\textbf{Quanto} & Usuário clica em um produto. \\ \hline
\textbf{Então} & Sistema redireciona para tela de detalhe. \\ \hline
\end{tabularx}

\begin{tabularx}{0.9\textwidth}{|l|X|}
\multicolumn{2}{@{}l}{\textbf{4. Deve permitir visualizar os produtos por mapa.}} \\ \hline
\textbf{Dado que} & Sistema carregou ao menos 1 produto. (R1) \\ \hline
\textbf{Quanto} & Usuário clica no ícone de mapa de um produto. \\ \hline
\textbf{Então} & Sistema apresenta a visualização por mapa. \\ \hline
\end{tabularx}

\begin{tabularx}{0.9\textwidth}{|l|X|}
\multicolumn{2}{@{}l}{\textbf{5. Deve permitir filtrar a busca.}} \\ \hline
\textbf{Dado que} & Usuário está na tela de listagem. \\ \hline
\textbf{Quanto} & Usuário clica no ícone de filtro. \\ \hline
\textbf{Então} & Sistema apresenta as opções de filtragem. \\ \hline
\end{tabularx}

\subsection*{\textbf{REGRAS DE NEGÓCIO DA HISTÓRIA}}

\begin{itemize}
    \item[] R1 - Produto cadastrado no banco de dados.
    \item[] R2 - Tamanho máximo do texto 256 caracteres.
\end{itemize}

\section{Visualizar mapa}%%%%%%%%%%%%%%%%%%%%%%%%%%%%%%%%%%%%%

\begin{tabular}{|ll|}
\hline
\multicolumn{2}{|c|}{\textbf{UC\nhist - \currentname}}    \\ \hline
\multicolumn{1}{|l|}{\textbf{Sendo}}     & um usuário \\ \hline
\multicolumn{1}{|l|}{\textbf{Quero}}     & visualizar os produtos buscados no mapa\\ \hline
\multicolumn{1}{|l|}{\textbf{Para}}      & verificar no mapa a localização do estabelecimento\\ \hline
\multicolumn{1}{|l|}{\textbf{Protótipo}} & 
\begin{minipage}{0.48\textwidth} 
\begin{figure}[H]
\caption{\label{fig:label} TELA MAPA}
\includegraphics[width=\textwidth]{fig/telas/t_mapa.png}
\footnotesize \centering
\par FONTE: O Autor (2024)
\end{figure}
\end{minipage}
 \\ \hline
\end{tabular}

\subsection*{\textbf{CRITÉRIOS DE ACEITAÇÃO}}

\begin{enumerate}[leftmargin=2cm]
    \item Deve definir a localização do usuário por GPS.
    \item Deve mostrar os produtos buscados em suas respectivas localizações no mapa.
    \item Deve permitir fazer uma nova busca.
    \item Deve permitir selecionar uma localização no mapa para mostrar mais detalhes.
\end{enumerate}

\subsection*{\textbf{CRITÉRIOS DE ACEITAÇÃO - DETALHAMENTO}}


\begin{tabularx}{0.9\textwidth}{|l|X|}
\multicolumn{2}{@{}l}{\textbf{1. Deve definir a localização do usuário por GPS.}} \\ \hline
\textbf{Dado que} & Usuário selecionou um produto. \\ \hline
\textbf{Quanto} & Sistema apresenta visualização por mapa. \\ \hline
\textbf{Então} & Sistema salva localização do usuário por ip. \\ \hline
\end{tabularx}

\begin{tabularx}{0.9\textwidth}{|l|X|}
\multicolumn{2}{@{}l}{\textbf{\makecell[l]{2. Deve mostrar os produtos buscados em suas respectivas \\localizações no mapa.}}} \\ \hline
\textbf{Dado que} & Usuário selecionou um produto. \\ \hline
\textbf{Quanto} & Usuário está na tela do mapa. \\ \hline
\textbf{Então} & Sistema mostra os produtos em seus respectivos locais no mapa. \\ \hline
\end{tabularx}

\begin{tabularx}{0.9\textwidth}{|l|X|}
\multicolumn{2}{@{}l}{\textbf{3. Deve permitir fazer uma nova busca}} \\ \hline
\textbf{Dado que} & Usuário inseriu o nome de um produto. (R2)\\ \hline
\textbf{Quanto} & Usuário clica em "Buscar". \\ \hline
\textbf{Então} & Sistema redireciona para tela de listagem. \\ \hline
\end{tabularx}

\begin{tabularx}{0.9\textwidth}{|l|X|}
\multicolumn{2}{@{}l}{\textbf{\makecell[l]{4. Deve permitir selecionar uma localização no mapa para mostrar \\mais detalhes.}}} \\ \hline
\textbf{Dado que} & Sistema encontrou 1 produto. (R1) \\ \hline
\textbf{Quanto} & Usuário seleciona um produto no mapa. \\ \hline
\textbf{Então} & Sistema redireciona para tela de detalhe. \\ \hline
\end{tabularx}

\subsection*{\textbf{REGRAS DE NEGÓCIO DA HISTÓRIA}}

\begin{itemize}
    \item[] R1 - Produto cadastrado no banco de dados.
    \item[] R2 - Tamanho máximo do texto 256 caracteres.
\end{itemize}


\section{Filtrar busca produtos}%%%%%%%%%%%%%%%%%%%%%%%%%%%%%%%%%%%%%

\begin{tabular}{|ll|}
\hline
\multicolumn{2}{|c|}{\textbf{UC\nhist - \currentname}}    \\ \hline
\multicolumn{1}{|l|}{\textbf{Sendo}}     & um usuário \\ \hline
\multicolumn{1}{|l|}{\textbf{Quero}}     & filtrar a minha busca\\ \hline
\multicolumn{1}{|l|}{\textbf{Para}}      & encontrar um produto especificando os parâmetros da busca\\ \hline
\multicolumn{1}{|l|}{\textbf{Protótipo}} & 
\begin{minipage}{0.48\textwidth} 
\begin{figure}[H]
\caption{\label{fig:label} TELA FILTRAR BUSCA}
\includegraphics[width=\textwidth]{fig/telas/t_filtros.jpg}
\footnotesize \centering
\par FONTE: O Autor (2024)
\end{figure}
\end{minipage}
 \\ \hline
\end{tabular}

\subsection*{\textbf{CRITÉRIOS DE ACEITAÇÃO}}

\begin{enumerate}[leftmargin=2cm]
    \item Deve permitir inserir o nome do produto.
    \item Deve permitir especificar a distância máxima.
    \item Deve permitir especificar o preço mínimo e máximo.
    \item Deve permitir especificar a localização do produto.
\end{enumerate}

\subsection*{\textbf{CRITÉRIOS DE ACEITAÇÃO - DETALHAMENTO}}


\begin{tabularx}{0.9\textwidth}{|l|X|}
\multicolumn{2}{@{}l}{\textbf{1. Deve permitir inserir o nome do produto.}} \\ \hline
\textbf{Dado que} & Usuário clicou no ícone de filtro. \\ \hline
\textbf{Quanto} & Usuário clica no campo "Produto". \\ \hline
\textbf{Então} & Sistema permite inserir nome de um produto. (R1)\\ \hline
\end{tabularx}

\begin{tabularx}{0.9\textwidth}{|l|X|}
\multicolumn{2}{@{}l}{\textbf{2. Deve permitir especificar a distância máxima.}} \\ \hline
\textbf{Dado que} & Usuário clicou no ícone de filtro. \\ \hline
\textbf{Quanto} & Usuário clica no campo "Distância". \\ \hline
\textbf{Então} & Sistema permite inserir distância. (R2) \\ \hline
\end{tabularx}

\begin{tabularx}{0.9\textwidth}{|l|X|}
\multicolumn{2}{@{}l}{\textbf{3. Deve permitir especificar o preço mínimo e máximo.}} \\ \hline
\textbf{Dado que} & Usuário clicou no ícone de filtro. \\ \hline
\textbf{Quanto} & Usuário clica no campo "Preço". \\ \hline
\textbf{Então} & Sistema permite inserir preço mínimo e máximo. (R3) \\ \hline
\end{tabularx}

\begin{tabularx}{0.9\textwidth}{|l|X|}
\multicolumn{2}{@{}l}{\textbf{4. Deve permitir especificar a localização do produto.}} \\ \hline
\textbf{Dado que} & Usuário clicou no ícone de filtro  \\ \hline
\textbf{Quanto} & Usuário clica no campo "Localização". \\ \hline
\textbf{Então} & Sistema permite inserir localização. (R4) \\ \hline
\end{tabularx}

\subsection*{\textbf{REGRAS DE NEGÓCIO DA HISTÓRIA}}

\begin{itemize}
    \item[] R1 - Tamanho máximo do texto 256 caracteres.
    \item[] R2 - Distância mínima 1km, distância máxima 15km.
    \item[] R3 - Preço mínimo 0, preço máximo 999,999.
    \item[] R4 - CEP válido ou localização no mapa.
\end{itemize}



\section{Detalhes produto}%%%%%%%%%%%%%%%%%%%%%%%%%%%%%%%%%%%%%

\begin{tabular}{|ll|}
\hline
\multicolumn{2}{|c|}{\textbf{UC\nhist - \currentname}}    \\ \hline
\multicolumn{1}{|l|}{\textbf{Sendo}}     & um usuário \\ \hline
\multicolumn{1}{|l|}{\textbf{Quero}}     & visualizar os detalhes do produto\\ \hline
\multicolumn{1}{|l|}{\textbf{Para}}      & analisar a localização e preço do produto\\ \hline
\multicolumn{1}{|l|}{\textbf{Protótipo}} & 
\begin{minipage}{0.48\textwidth} 
\begin{figure}[H]
\caption{\label{fig:label} TELA DETALHES PRODUTO}
\includegraphics[width=\textwidth]{fig/telas/t_produto.jpg}
\footnotesize \centering
\par FONTE: O Autor (2024)
\end{figure}
\end{minipage}
 \\ \hline
\end{tabular}

\subsection*{\textbf{CRITÉRIOS DE ACEITAÇÃO}}

\begin{enumerate}[leftmargin=2cm]
    \item Deve mostrar as informações do produto como: foto, nome, estabelecimento, preço e localização.
    \item Deve mostrar outros estabelecimentos com o mesmo produto ordenado por distância.
    \item Deve permitir mostrar o estabelecimento no mapa.
    \item Deve permitir sugerir um novo preço para o produto no estabelecimento.
\end{enumerate}

\subsection*{\textbf{CRITÉRIOS DE ACEITAÇÃO - DETALHAMENTO}}
\textbf{Critério de contexto} (Válido como premissa para todos os critérios):

\begin{tabularx}{0.9\textwidth}{@{}l X }
 \textbf{Dado que} & Sistema buscou ao menos 1 produto. \\ 
 \textbf{E} & Usuário clicou em 1 produto.
\end{tabularx}


\begin{tabularx}{0.9\textwidth}{|l|X|}
\multicolumn{2}{@{}l}{\textbf{\makecell[l]{1. Deve mostrar as informações do produto como: foto, nome, \\estabelecimento, preço e localização.}}} \\ \hline
\textbf{Dado que} & Usuário está na tela de detalhe. \\ \hline
\textbf{Quanto} & Sistema carrega informações do produto. \\ \hline
\textbf{Então} & Sistema apresenta informações do produto. \\ \hline
\end{tabularx}

\begin{tabularx}{0.9\textwidth}{|l|X|}
\multicolumn{2}{@{}l}{\textbf{\makecell[l]{2. Deve mostrar outros estabelecimentos com o mesmo produto \\ordenado por distância.}}} \\ \hline
\textbf{Dado que} & Usuário está na tela de detalhe. \\ \hline
\textbf{Quanto} & Sistema carrega informações do produto. \\ \hline
\textbf{Então} & Sistema apresenta outras lojas com o mesmo produto. \\ \hline
\end{tabularx}

\begin{tabularx}{0.9\textwidth}{|l|X|}
\multicolumn{2}{@{}l}{\textbf{3. Deve permitir mostrar o estabelecimento no mapa.}} \\ \hline
\textbf{Dado que} & Usuário está na tela de detalhe. \\ \hline
\textbf{Quanto} & Usuário clica no ícone do mapa. \\ \hline
\textbf{Então} & Sistema apresenta a localização do produto no mapa. \\ \hline
\end{tabularx}

\begin{tabularx}{0.9\textwidth}{|l|X|}
\multicolumn{2}{@{}l}{\textbf{\makecell[l]{4. Deve permitir sugerir um novo preço para o produto no \\estabelecimento.}}} \\ \hline
\textbf{Dado que} & Usuário está na tela de detalhe. \\ \hline
\textbf{Quanto} & Usuário clica no botão de sugestão. \\ \hline
\textbf{Então} & Sistema apresenta formulário de sugestão de preço. \\ \hline
\end{tabularx}


\section{Sugerir edição}%%%%%%%%%%%%%%%%%%%%%%%%%%%%%%%%%%%%%

\begin{tabular}{|ll|}
\hline
\multicolumn{2}{|c|}{\textbf{UC\nhist - \currentname}}    \\ \hline
\multicolumn{1}{|l|}{\textbf{Sendo}}     & um usuário \\ \hline
\multicolumn{1}{|l|}{\textbf{Quero}}     & sugerir a edição de preço de um produto\\ \hline
\multicolumn{1}{|l|}{\textbf{Para}}      & para que o produto apareça com o novo preço atualizado\\ \hline
\multicolumn{1}{|l|}{\textbf{Protótipo}} & 
\begin{minipage}{0.48\textwidth} 
\begin{figure}[H]
\caption{\label{fig:label} TELA SUGERIR EDIÇÃO}
\includegraphics[width=\textwidth]{fig/telas/t_sugerir.jpg}
\footnotesize \centering
\par FONTE: O Autor (2024)
\end{figure}
\end{minipage}
 \\ \hline
\end{tabular}

\subsection*{\textbf{CRITÉRIOS DE ACEITAÇÃO}}

\begin{enumerate}[leftmargin=2cm]
    \item Deve mostrar as informações do produto.
    \item Deve permitir inserir um novo preço.
    \item Deve permitir sugerir um novo preço.
\end{enumerate}

\subsection*{\textbf{CRITÉRIOS DE ACEITAÇÃO - DETALHAMENTO}}
\textbf{Critério de contexto} (Válido como premissa para todos os critérios):

\begin{tabularx}{0.9\textwidth}{@{}l X }
\textbf{Dado que} & Sistema buscou ao menos 1 produto.\\ 
\textbf{E} & Usuário clicou em 1 produto.\\
\textbf{E} & Usuário clicou no botão de sugestão.
\end{tabularx}


\begin{tabularx}{0.9\textwidth}{|l|X|}
\multicolumn{2}{@{}l}{\textbf{1. Deve mostrar as informações do produto.}} \\ \hline
\textbf{Dado que} & Usuário está na tela de sugestão. \\ \hline
\textbf{Quanto} & Sistema carrega informações do produto. \\ \hline
\textbf{Então} & Sistema apresenta informações atuais do produto. \\ \hline
\end{tabularx}

\begin{tabularx}{0.9\textwidth}{|l|X|}
\multicolumn{2}{@{}l}{\textbf{2. Deve permitir inserir um novo preço.}} \\ \hline
\textbf{Dado que} & Usuário está na tela de sugestão. \\ \hline
\textbf{Quanto} & Usuário clica no campo "Novo preço". \\ \hline
\textbf{Então} & Sistema permite inserir novo preço. (R1) \\ \hline
\end{tabularx}

\begin{tabularx}{0.9\textwidth}{|l|X|}
\multicolumn{2}{@{}l}{\textbf{3. Deve permitir sugerir um novo preço.}} \\ \hline
\textbf{Dado que} & Usuário inseriu preço válido. (R1) \\ \hline
\textbf{Quanto} & Usuário clica no botão "Editar". \\ \hline
\textbf{Então} & Sistema cadastra nova sugestão. \\ \hline
\end{tabularx}

\subsection*{\textbf{REGRAS DE NEGÓCIO DA HISTÓRIA}}

\begin{itemize}
    \item[] R1 - Preço mínimo 0, preço máximo 999,999.
\end{itemize}


\section{Avaliar sugestão}%%%%%%%%%%%%%%%%%%%%%%%%%%%%%%%%%%%%%

\begin{tabular}{|ll|}
\hline
\multicolumn{2}{|c|}{\textbf{UC\nhist - \currentname}}    \\ \hline
\multicolumn{1}{|l|}{\textbf{Sendo}}     & um usuário \\ \hline
\multicolumn{1}{|l|}{\textbf{Quero}}     & avaliar uma sugestão de novo preço\\ \hline
\multicolumn{1}{|l|}{\textbf{Para}}      & para que o produto fique com o preço atualizado\\ \hline
\multicolumn{1}{|l|}{\textbf{Protótipo}} & 
\begin{minipage}{0.48\textwidth} 
\begin{figure}[H]
\caption{\label{fig:label} TELA AVALIAR SUGESTÃO}
\includegraphics[width=\textwidth]{fig/telas/t_avaliar.jpg}
\footnotesize \centering
\par FONTE: O Autor (2024)
\end{figure}
\end{minipage}
 \\ \hline
\end{tabular}

\subsection*{\textbf{CRITÉRIOS DE ACEITAÇÃO}}

\begin{enumerate}[leftmargin=2cm]
    \item Deve mostrar as informações do produto atual e o novo preço sugerido.
    \item Deve permitir avaliar a sugestão com um positivo ou negativo.
    \item Deve atualizar o preço do produto se maioria das avaliações forem positivas.
\end{enumerate}

\subsection*{\textbf{CRITÉRIOS DE ACEITAÇÃO - DETALHAMENTO}}
\textbf{Critério de contexto} (Válido como premissa para todos os critérios):

\begin{tabularx}{0.9\textwidth}{@{}l X }
\textbf{Dado que} & Sistema buscou ao menos 1 produto. \\ 
\textbf{E} & Usuário clicou em 1 produto.
\end{tabularx}


\begin{tabularx}{0.9\textwidth}{|l|X|}
\multicolumn{2}{@{}l}{\textbf{\makecell[l]{1. Deve mostrar as informações do produto atual e o novo preço \\sugerido.}}} \\ \hline
\textbf{Dado que} & Usuário está na tela de avaliação. \\ \hline
\textbf{Quanto} & Sistema carrega informações do produto. \\ \hline
\textbf{Então} & Sistema apresenta informações da sugestão. \\ \hline
\end{tabularx}

\begin{tabularx}{0.9\textwidth}{|l|X|}
\multicolumn{2}{@{}l}{\textbf{2. Deve permitir avaliar a sugestão com um positivo ou negativo.}} \\ \hline
\textbf{Dado que} & Usuário está na tela de avaliação. \\ \hline
\textbf{Quanto} & Usuário avalia uma sugestão. \\ \hline
\textbf{Então} & Sistema atualiza números de avaliações. \\ \hline
\end{tabularx}

\begin{tabularx}{0.9\textwidth}{|l|X|}
\multicolumn{2}{@{}l}{\textbf{\makecell[l]{3. Deve atualizar o preço do produto se maioria das avaliações forem \\positivas.}}} \\ \hline
\textbf{Dado que} & Usuário está na tela de avaliação.  \\ \hline
\textbf{Quanto} & Usuário avalia uma sugestão. \\ \hline
\textbf{Então} & Sistema atualiza preço do produto. (R1) \\ \hline
\end{tabularx}

\subsection*{\textbf{REGRAS DE NEGÓCIO DA HISTÓRIA}}

\begin{itemize}
    \item[] R1 - Mais de 75\% das avaliações positivas.
\end{itemize}


\section{Cadastrar produto}%%%%%%%%%%%%%%%%%%%%%%%%%%%%%%%%%%%%%

\begin{tabular}{|ll|}
\hline
\multicolumn{2}{|c|}{\textbf{UC\nhist - \currentname}}    \\ \hline
\multicolumn{1}{|l|}{\textbf{Sendo}}     & um usuário \\ \hline
\multicolumn{1}{|l|}{\textbf{Quero}}     & cadastrar um novo produto\\ \hline
\multicolumn{1}{|l|}{\textbf{Para}}      & para que o produto apareça no sistema\\ \hline
\multicolumn{1}{|l|}{\textbf{Protótipo}} & 
\begin{minipage}{0.48\textwidth} 
\begin{figure}[H]
\caption{\label{fig:label} TELA CADASTRAR PRODUTO}
\includegraphics[width=\textwidth]{fig/telas/t_novproduto.jpg}
\footnotesize \centering
\par FONTE: O Autor (2024)
\end{figure}
\end{minipage}
 \\ \hline
\end{tabular}

\subsection*{\textbf{CRITÉRIOS DE ACEITAÇÃO}}

\begin{enumerate}[leftmargin=2cm]
    \item Deve permitir inserir o nome do produto, estabelecimento e preço do produto.
    \item Deve cadastrar o produto.
    \item Deve criar uma sugestão de novo preço pro produto caso já esteja cadastrado.
\end{enumerate}

\subsection*{\textbf{CRITÉRIOS DE ACEITAÇÃO - DETALHAMENTO}}
\textbf{Critério de contexto} (Válido como premissa para todos os critérios):

\begin{tabularx}{0.9\textwidth}{@{}l X }
 \textbf{Dado que} & Usuário esta logado \\ 
 \textbf{E} & Clicou no botão "Cadastrar produto".
\end{tabularx}

\begin{tabularx}{0.9\textwidth}{|l|X|}
\multicolumn{2}{@{}l}{\textbf{\makecell[l]{1. Deve permitir inserir o nome do produto, estabelecimento e \\preço do produto.}}} \\ \hline
\textbf{Dado que} & Usuário na tela de cadastro. \\ \hline
\textbf{Quanto} & Usuário clica em algum campo. \\ \hline
\textbf{Então} & Sistema permite edição. \\ \hline
\end{tabularx}

\begin{tabularx}{0.9\textwidth}{|l|X|}
\multicolumn{2}{@{}l}{\textbf{2. Deve cadastrar o produto.}} \\ \hline
\textbf{Dado que} & Usuário inseriu dados corretamente. (R1) (R2) \\ \hline
\textbf{Quanto} & Usuário clica no botão "Cadastrar". \\ \hline
\textbf{Então} & Sistema cadastra produto. \\ \hline
\end{tabularx}

\begin{tabularx}{0.9\textwidth}{|l|X|}
\multicolumn{2}{@{}l}{\textbf{\makecell[l]{3. Deve criar uma sugestão de novo preço pro produto caso já esteja \\cadastrado.}}} \\ \hline
\textbf{Dado que} & Usuário inseriu dados corretamente. (R1) (R2) \\ \hline
\textbf{Quanto} & Produto já cadastrado no estabelecimento. \\ \hline
\textbf{Então} & Sistema cadastra sugestão de preço. \\ \hline
\end{tabularx}

\subsection*{\textbf{REGRAS DE NEGÓCIO DA HISTÓRIA}}

\begin{itemize}
    \item[] R1 - Localização correta no mapa.
    \item[] R2 - Preço mínimo 0, preço máximo 999,999.
\end{itemize}


\section{Digitalizar nota}%%%%%%%%%%%%%%%%%%%%%%%%%%%%%%%%%%%%%

\begin{tabular}{|ll|}
\hline
\multicolumn{2}{|c|}{\textbf{UC\nhist - \currentname}}    \\ \hline
\multicolumn{1}{|l|}{\textbf{Sendo}}     & um usuário \\ \hline
\multicolumn{1}{|l|}{\textbf{Quero}}     & digitalizar uma nota\\ \hline
\multicolumn{1}{|l|}{\textbf{Para}}      & para inserir automaticamente os produtos no sistema\\ \hline
\multicolumn{1}{|l|}{\textbf{Protótipo}} & 
\begin{minipage}{0.48\textwidth} 
\begin{figure}[H]
\caption{\label{fig:label} TELA DIGITALIZAR}
\includegraphics[width=\textwidth]{fig/telas/t_digitaliza.jpg}
\footnotesize \centering
\par FONTE: O Autor (2024)
\end{figure}
\end{minipage}
 \\ \hline
\end{tabular}

\subsection*{\textbf{CRITÉRIOS DE ACEITAÇÃO}}

\begin{enumerate}[leftmargin=2cm]
    \item Deve permitir tirar uma foto de uma nota fiscal.
    \item Deve digitalizar os itens da nota em texto.
\end{enumerate}

\subsection*{\textbf{CRITÉRIOS DE ACEITAÇÃO - DETALHAMENTO}}
\textbf{Critério de contexto} (Válido como premissa para todos os critérios):

\begin{tabularx}{0.9\textwidth}{@{}l X }
\textbf{Dado que} & Usuário esta logado \\ 
\textbf{E} & Clicou no botão "Cadastrar produto".\\
\textbf{E} & Clicou no ícone de nota fiscal.
\end{tabularx}


\begin{tabularx}{0.9\textwidth}{|l|X|}
\multicolumn{2}{@{}l}{\textbf{1. Deve permitir tirar uma foto de uma nota fiscal.}} \\ \hline
\textbf{Dado que} & Usuário está na tela de nota fiscal. \\ \hline
\textbf{Quanto} & Usuário clica no ícone de camera. \\ \hline
\textbf{Então} & Sistema permite escanear uma foto. \\ \hline
\end{tabularx}

\begin{tabularx}{0.9\textwidth}{|l|X|}
\multicolumn{2}{@{}l}{\textbf{2. Deve digitalizar os itens da nota em texto.}} \\ \hline
\textbf{Dado que} & Usuário escaneou nota corretamente. (R1) \\ \hline
\textbf{Quanto} & Usuário clica no ícone de camera. \\ \hline
\textbf{Então} & Sistema redireciona para tela de confirmação. \\ \hline
\end{tabularx}

\subsection*{\textbf{REGRAS DE NEGÓCIO DA HISTÓRIA}}

\begin{itemize}
    \item[] R1 - QR Code da nota fiscal encontrada.
\end{itemize}

\section{Confirmar digitalização}%%%%%%%%%%%%%%%%%%%%%%%%%%%%%%%%%%%%%

\begin{tabular}{|ll|}
\hline
\multicolumn{2}{|c|}{\textbf{UC\nhist - \currentname}}    \\ \hline
\multicolumn{1}{|l|}{\textbf{Sendo}}     & um usuário \\ \hline
\multicolumn{1}{|l|}{\textbf{Quero}}     & confirmar a digitalização\\ \hline
\multicolumn{1}{|l|}{\textbf{Para}}      & para corrigir eventuais erros inseridos pelo sistema\\ \hline
\multicolumn{1}{|l|}{\textbf{Protótipo}} & 
\begin{minipage}{0.48\textwidth} 
\begin{figure}[H]
\caption{\label{fig:label} TELA DIGITALIZAÇÃO}
\includegraphics[width=\textwidth]{fig/telas/t_confdigitaliza.jpg}
\footnotesize \centering
\par FONTE: O Autor (2024)
\end{figure}
\end{minipage}
 \\ \hline
\end{tabular}

\subsection*{\textbf{CRITÉRIOS DE ACEITAÇÃO}}

\begin{enumerate}[leftmargin=2cm]
    \item Deve permitir inserir o estabelecimento da nota.
    \item Deve permitir alterar os produtos digitalizados pelo sistema.
    \item Deve permitir adicionar mais produtos.
    \item Deve cadastrar os produtos.
    \item Deve criar uma sugestão de novo preço pro produto caso já esteja cadastrado.
\end{enumerate}

\subsection*{\textbf{CRITÉRIOS DE ACEITAÇÃO - DETALHAMENTO}}
\textbf{Critério de contexto} (Válido como premissa para todos os critérios):

\begin{tabularx}{0.9\textwidth}{@{}l X }
\textbf{Dado que} & Usuário esta logado \\ 
\textbf{E} & Clicou no botão "Cadastrar produto".\\
\textbf{E} & Clicou no ícone de nota fiscal.\\
\textbf{E} & Escaneou uma nota válida.
\end{tabularx}


\begin{tabularx}{0.9\textwidth}{|l|X|}
\multicolumn{2}{@{}l}{\textbf{1. Deve permitir inserir o estabelecimento da nota.}} \\ \hline
\textbf{Dado que} & Usuário está na tela de confirmação. \\ \hline
\textbf{Quanto} & Usuário clica no campo "Estabelecimento". \\ \hline
\textbf{Então} & Sistema permite selecionar no mapa a localiação. \\ \hline
\end{tabularx}

\begin{tabularx}{0.9\textwidth}{|l|X|}
\multicolumn{2}{@{}l}{\textbf{2. Deve permitir alterar os produtos digitalizados pelo sistema.}} \\ \hline
\textbf{Dado que} & Usuário está na tela de confirmação. \\ \hline
\textbf{Quanto} & Usuário clica no ícone de edição. \\ \hline
\textbf{Então} & Sistema apresenta formulário de edição. \\ \hline
\end{tabularx}

\begin{tabularx}{0.9\textwidth}{|l|X|}
\multicolumn{2}{@{}l}{\textbf{3. Deve permitir adicionar mais produtos.}} \\ \hline
\textbf{Dado que} & Usuário está na tela de confirmação. \\ \hline
\textbf{Quanto} & Usuário clica no ícone de adição. \\ \hline
\textbf{Então} & Sistema apresenta formulário de adição. \\ \hline
\end{tabularx}

\begin{tabularx}{0.9\textwidth}{|l|X|}
\multicolumn{2}{@{}l}{\textbf{4. Deve cadastrar os produtos.}} \\ \hline
\textbf{Dado que} & Usuário está na tela de confirmação. \\ \hline
\textbf{Quanto} & Usuário clica no ícone "Confirmar". \\ \hline
\textbf{Então} & Sistema cadastra produtos. \\ \hline
\end{tabularx}

\begin{tabularx}{0.9\textwidth}{|l|X|}
\multicolumn{2}{@{}l}{\textbf{\makecell[l]{5. Deve criar uma sugestão de novo preço pro produto caso já esteja \\cadastrado.}}} \\ \hline
\textbf{Dado que} & Usuário clica no ícone "Confirmar".\\ \hline
\textbf{Quanto} & Produto já cadastrado. \\ \hline
\textbf{Então} & Sistema cadastra sugestão de novo preço. \\ \hline
\end{tabularx}


\section{Menu perfil}%%%%%%%%%%%%%%%%%%%%%%%%%%%%%%%%%%%%%

\begin{tabular}{|ll|}
\hline
\multicolumn{2}{|c|}{\textbf{UC\nhist - \currentname}}    \\ \hline
\multicolumn{1}{|l|}{\textbf{Sendo}}     & um usuário \\ \hline
\multicolumn{1}{|l|}{\textbf{Quero}}     & visualizar os detalhes do meu perfil\\ \hline
\multicolumn{1}{|l|}{\textbf{Para}}      & alterar minha senha ou foto\\ \hline
\multicolumn{1}{|l|}{\textbf{Protótipo}} & 
\begin{minipage}{0.48\textwidth} 
\begin{figure}[H]
\caption{\label{fig:label} TELA PERFIL}
\includegraphics[width=\textwidth]{fig/telas/t_mperfil.jpg}
\footnotesize \centering
\par FONTE: O Autor (2024)
\end{figure}
\end{minipage}
 \\ \hline
\end{tabular}

\subsection*{\textbf{CRITÉRIOS DE ACEITAÇÃO}}

\begin{enumerate}[leftmargin=2cm]
    \item Deve mostrar as informações do usuário logado.
    \item Deve permitir alterar a foto.
    \item Deve permitir alterar a senha.
\end{enumerate}

\subsection*{\textbf{CRITÉRIOS DE ACEITAÇÃO - DETALHAMENTO}}
\textbf{Critério de contexto} (Válido como premissa para todos os critérios):

\begin{tabularx}{0.9\textwidth}{@{}l X }
 \textbf{Dado que} & Usuário logado. \\ 
 \textbf{E} & Clicou no menu lateral.
\end{tabularx}


\begin{tabularx}{0.9\textwidth}{|l|X|}
\multicolumn{2}{@{}l}{\textbf{1. Deve mostrar as informações do usuário logado.}} \\ \hline
\textbf{Dado que} & Usuário está na tela de perfil. \\ \hline
\textbf{Quanto} & Sistema carrega informações do usuário. \\ \hline
\textbf{Então} & Sistema apresenta informações do usuário. \\ \hline
\end{tabularx}

\begin{tabularx}{0.9\textwidth}{|l|X|}
\multicolumn{2}{@{}l}{\textbf{2. Deve permitir alterar a foto.}} \\ \hline
\textbf{Dado que} & Usuário está na tela de perfil. \\ \hline
\textbf{Quanto} & Usuário clica no ícone de camera. \\ \hline
\textbf{Então} & Sistema permite atualizar foto. (R1) \\ \hline
\end{tabularx}

\begin{tabularx}{0.9\textwidth}{|l|X|}
\multicolumn{2}{@{}l}{\textbf{3. Deve permitir alterar a senha.}} \\ \hline
\textbf{Dado que} & Usuário está na tela de perfil. \\ \hline
\textbf{Quanto} & Usuário clica em "Alterar senha". \\ \hline
\textbf{Então} & Sistema redireciona para tela de alteração. \\ \hline
\end{tabularx}

\subsection*{\textbf{REGRAS DE NEGÓCIO DA HISTÓRIA}}

\begin{itemize}
    \item[] R1 - Tamanho máximo do arquivo 5mb.
\end{itemize}



\section{Alterar senha}%%%%%%%%%%%%%%%%%%%%%%%%%%%%%%%%%%%%%

\begin{tabular}{|ll|}
\hline
\multicolumn{2}{|c|}{\textbf{UC\nhist - \currentname}}    \\ \hline
\multicolumn{1}{|l|}{\textbf{Sendo}}     & um usuário \\ \hline
\multicolumn{1}{|l|}{\textbf{Quero}}     & alterar minha senha\\ \hline
\multicolumn{1}{|l|}{\textbf{Para}}      & alterar minha senha atual para uma nova\\ \hline
\multicolumn{1}{|l|}{\textbf{Protótipo}} & 
\begin{minipage}{0.48\textwidth} 
\begin{figure}[H]
\caption{\label{fig:label} TELA ALTERAR SENHA}
\includegraphics[width=\textwidth]{fig/telas/t_altsenha.jpg}
\footnotesize \centering
\par FONTE: O Autor (2024)
\end{figure}
\end{minipage}
 \\ \hline
\end{tabular}

\subsection*{\textbf{CRITÉRIOS DE ACEITAÇÃO}}

\begin{enumerate}[leftmargin=2cm]
    \item Deve permitir inserir a senha atual e uma nova senha.
    \item Deve permitir alterar a senha caso a senha atual seja inserida corretamente.
\end{enumerate}

\subsection*{\textbf{CRITÉRIOS DE ACEITAÇÃO - DETALHAMENTO}}
\textbf{Critério de contexto} (Válido como premissa para todos os critérios):

\begin{tabularx}{0.9\textwidth}{@{}l X }
\textbf{Dado que} & Usuário logado. \\ 
\textbf{E} & Clicou no menu lateral.\\
\textbf{E} & Clicou em alterar senha.
\end{tabularx}


\begin{tabularx}{0.9\textwidth}{|l|X|}
\multicolumn{2}{@{}l}{\textbf{1. Deve permitir inserir a senha atual e uma nova senha.}} \\ \hline
\textbf{Dado que} & Usuário está na tela de alterar senha. \\ \hline
\textbf{Quanto} & Usuário clica nos campos de senha. \\ \hline
\textbf{Então} & Sistema permite inserir dado. \\ \hline
\end{tabularx}

\begin{tabularx}{0.9\textwidth}{|l|X|}
\multicolumn{2}{@{}l}{\textbf{\makecell[l]{2. Deve permitir alterar a senha caso a senha atual seja inserida \\corretamente.}}} \\ \hline
\textbf{Dado que} & Usuário inseriu campos corretamente. (R1) (R2) (R3) \\ \hline
\textbf{Quanto} & Usuário clicou em "Alterar". \\ \hline
\textbf{Então} & Sistema atualiza senha do usuário. \\ \hline
\end{tabularx}

\subsection*{\textbf{REGRAS DE NEGÓCIO DA HISTÓRIA}}

\begin{itemize}
    \item[] R1 - Senha atual válida.
    \item[] R2 - Novas e confirmação iguais.
    \item[] R3 - Tamanho máximo do texto 256 caracteres.
\end{itemize}




\section{Logar no sistema}%%%%%%%%%%%%%%%%%%%%%%%%%%%%%%%%%%%%%

\begin{tabular}{|ll|}
\hline
\multicolumn{2}{|c|}{\textbf{UC\nhist - \currentname}}    \\ \hline
\multicolumn{1}{|l|}{\textbf{Sendo}}     & um usuário \\ \hline
\multicolumn{1}{|l|}{\textbf{Quero}}     & logar no sistema\\ \hline
\multicolumn{1}{|l|}{\textbf{Para}}      & poder avaliar e sugerir novos preços para os produtos\\ \hline
\multicolumn{1}{|l|}{\textbf{Protótipo}} & 
\begin{minipage}{0.48\textwidth} 
\begin{figure}[H]
\caption{\label{fig:label} TELA LOGIN}
\includegraphics[width=\textwidth]{fig/telas/t_login.jpg}
\footnotesize \centering
\par FONTE: O Autor (2024)
\end{figure}
\end{minipage}
 \\ \hline
\end{tabular}

\subsection*{\textbf{CRITÉRIOS DE ACEITAÇÃO}}

\begin{enumerate}[leftmargin=2cm]
    \item Deve permitir inserir email e senha.
    \item Deve permitir fazer cadastro pela conta Google ou Facebook.
    \item Deve permitir recuperar senha.
    \item Deve permitir acessar formulário de cadastro.
    \item Deve permitir logar no sistema.
\end{enumerate}

\subsection*{\textbf{CRITÉRIOS DE ACEITAÇÃO - DETALHAMENTO}}
\textbf{Critério de contexto} (Válido como premissa para todos os critérios):

\begin{tabularx}{0.9\textwidth}{@{}l X }
 \textbf{Dado que} & Usuário clicou no botão "Login". \\ 
\end{tabularx}


\begin{tabularx}{0.9\textwidth}{|l|X|}
\multicolumn{2}{@{}l}{\textbf{1. Deve permitir inserir email e senha.}} \\ \hline
\textbf{Dado que} & Usuário está na tela de login. \\ \hline
\textbf{Quanto} & Usuário clica nos campos. \\ \hline
\textbf{Então} & Sistema permite inserir dado. (R1) (R2) \\ \hline
\end{tabularx}

\begin{tabularx}{0.9\textwidth}{|l|X|}
\multicolumn{2}{@{}l}{\textbf{2. Deve permitir fazer cadastro pela conta Google ou Facebook.}} \\ \hline
\textbf{Dado que} & Usuário está na tela de login. \\ \hline
\textbf{Quanto} & Usuário clica no botão Google ou Facebook \\ \hline
\textbf{Então} & Sistema redireciona para login externo. \\ \hline
\end{tabularx}

\begin{tabularx}{0.9\textwidth}{|l|X|}
\multicolumn{2}{@{}l}{\textbf{3. Deve permitir recuperar senha.}} \\ \hline
\textbf{Dado que} & Usuário está na tela de login. \\ \hline
\textbf{Quanto} & Usuário clica em "Esqueci minha senha". \\ \hline
\textbf{Então} & Sistema redireciona para tela de recuperação. \\ \hline
\end{tabularx}

\begin{tabularx}{0.9\textwidth}{|l|X|}
\multicolumn{2}{@{}l}{\textbf{4. Deve permitir acessar formulário de cadastro.}} \\ \hline
\textbf{Dado que} & Usuário está na tela de login. \\ \hline
\textbf{Quanto} & Usuário clica em "Criar conta". \\ \hline
\textbf{Então} & Sistema redireciona para tela de cadastro. \\ \hline
\end{tabularx}

\begin{tabularx}{0.9\textwidth}{|l|X|}
\multicolumn{2}{@{}l}{\textbf{5. Deve permitir logar no sistema.}} \\ \hline
\textbf{Dado que} & Usuário inseriu credenciais corretas. (R3) \\ \hline
\textbf{Quanto} & Usuário clica em "Entrar". \\ \hline
\textbf{Então} & Sistema autentica usuário e redireciona para tela inicial. \\ \hline
\end{tabularx}

\subsection*{\textbf{REGRAS DE NEGÓCIO DA HISTÓRIA}}

\begin{itemize}
    \item[] R1 - Tamanho maximo do texto 256 caracteres.
    \item[] R2 - Email válido.
    \item[] R3 - Usuário cadastrado no banco de dados.
\end{itemize}

\section{Criar conta}%%%%%%%%%%%%%%%%%%%%%%%%%%%%%%%%%%%%%

\begin{tabular}{|ll|}
\hline
\multicolumn{2}{|c|}{\textbf{UC\nhist - \currentname}}    \\ \hline
\multicolumn{1}{|l|}{\textbf{Sendo}}     & um usuário \\ \hline
\multicolumn{1}{|l|}{\textbf{Quero}}     & cadastrar no sistema\\ \hline
\multicolumn{1}{|l|}{\textbf{Para}}      & criar uma conta para logar no sistema\\ \hline
\multicolumn{1}{|l|}{\textbf{Protótipo}} & 
\begin{minipage}{0.48\textwidth} 
\begin{figure}[H]
\caption{\label{fig:label} TELA CADASTRO}
\includegraphics[width=\textwidth]{fig/telas/t_cadastro.jpg}
\footnotesize \centering
\par FONTE: O Autor (2024)
\end{figure}
\end{minipage}
 \\ \hline
\end{tabular}

\subsection*{\textbf{CRITÉRIOS DE ACEITAÇÃO}}

\begin{enumerate}[leftmargin=2cm]
    \item Deve permitir inserir email, nome e senha.
    \item Deve fazer cadastro no sistema.
\end{enumerate}

\subsection*{\textbf{CRITÉRIOS DE ACEITAÇÃO - DETALHAMENTO}}
\textbf{Critério de contexto} (Válido como premissa para todos os critérios):

\begin{tabularx}{0.9\textwidth}{@{}l X }
\textbf{Dado que} & Usuário clicou no botão "Criar conta". \\ 
\end{tabularx}


\begin{tabularx}{0.9\textwidth}{|l|X|}
\multicolumn{2}{@{}l}{\textbf{1. Deve permitir inserir email, nome e senha.}} \\ \hline
\textbf{Dado que} & Usuário está na tela de cadastro. \\ \hline
\textbf{Quanto} & Usuário clica nos campos. \\ \hline
\textbf{Então} & Sistema permite inserir dado. \\ \hline
\end{tabularx}

\begin{tabularx}{0.9\textwidth}{|l|X|}
\multicolumn{2}{@{}l}{\textbf{2. Deve fazer cadastro no sistema.}} \\ \hline
\textbf{Dado que} & Usuário preencheu corretamente. (R1) (R2) (R3) (R4) \\ \hline
\textbf{Quanto} & Usuário clica em "Cadastrar". \\ \hline
\textbf{Então} & Sistema cadastra novo usuário. \\ \hline
\end{tabularx}

\subsection*{\textbf{REGRAS DE NEGÓCIO DA HISTÓRIA}}

\begin{itemize}
    \item[] R1 - Email válido.
    \item[] R2 - Tamanho máximo do texto 256 caracteres.
    \item[] R3 - Senha e confirmação iguais.
    \item[] R4 - Usuário ainda não cadastrado.
\end{itemize}


\section{Recuperar senha}%%%%%%%%%%%%%%%%%%%%%%%%%%%%%%%%%%%%%

\begin{tabular}{|ll|}
\hline
\multicolumn{2}{|c|}{\textbf{UC\nhist - \currentname}}    \\ \hline
\multicolumn{1}{|l|}{\textbf{Sendo}}     & um usuário \\ \hline
\multicolumn{1}{|l|}{\textbf{Quero}}     & recuperar minha senha\\ \hline
\multicolumn{1}{|l|}{\textbf{Para}}      & acessar o sistema\\ \hline
\multicolumn{1}{|l|}{\textbf{Protótipo}} & 
\begin{minipage}{0.48\textwidth} 
\begin{figure}[H]
\caption{\label{fig:label} TELA RECUPERAR SENHA}
\includegraphics[width=\textwidth]{fig/telas/t_esqueci.jpg}
\footnotesize \centering
\par FONTE: O Autor (2024)
\end{figure}
\end{minipage}
 \\ \hline
\end{tabular}

\subsection*{\textbf{CRITÉRIOS DE ACEITAÇÃO}}

\begin{enumerate}[leftmargin=2cm]
    \item Deve permitir inserir email.
    \item Deve enviar nova senha gerada aleatoriamente para o email cadastrado.
\end{enumerate}

\subsection*{\textbf{CRITÉRIOS DE ACEITAÇÃO - DETALHAMENTO}}
\textbf{Critério de contexto} (Válido como premissa para todos os critérios):

\begin{tabularx}{0.9\textwidth}{@{}l X }
\textbf{Dado que} & Usuário clicou no botão "Esqueci minha senha". \\ 
\end{tabularx}


\begin{tabularx}{0.9\textwidth}{|l|X|}
\multicolumn{2}{@{}l}{\textbf{1. Deve permitir inserir email.}} \\ \hline
\textbf{Dado que} & Usuário está na tela de recuperação. \\ \hline
\textbf{Quanto} & Usuário clica no campo "Email". \\ \hline
\textbf{Então} & Sistema permite inserir email. \\ \hline
\end{tabularx}

\begin{tabularx}{0.9\textwidth}{|l|X|}
\multicolumn{2}{@{}l}{\textbf{\makecell[l]{2. Deve enviar nova senha gerada aleatoriamente para o email \\cadastrado.}}} \\ \hline
\textbf{Dado que} & Usuário preencheu corretamente. (R1) (R2) \\ \hline
\textbf{Quanto} & Usuário clica em "Recuperar". \\ \hline
\textbf{Então} & Sistema envia email com nova senha aleatória. \\ \hline
\end{tabularx}

\subsection*{\textbf{REGRAS DE NEGÓCIO DA HISTÓRIA}}

\begin{itemize}
    \item[] R1 - Email válido.
    \item[] R2 - Usuário cadastrado.
\end{itemize}
