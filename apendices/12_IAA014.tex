\label{ap:ap12}
\chapter{Gestão de Projetos de IA}
\section*{\textbf{A - ENUNCIADO}}

\subsection*{\textbf{1 Objetivo}}

\textcolor{black}{Individualmente, ler e resumir – seguindo o }\textit{\textcolor{black}{template }}\textcolor{black}{
fornecido – }\textbf{\textcolor{black}{um }}\textcolor{black}{ dos artigos abaixo:}



\foreignlanguage{english}{\textcolor{black}{AHMAD, L.; ABDELRAZEK, M.; ARORA, C.; BANO, M; GRUNDY, J. Requirements
practices and gaps when engineering human-centered Artificial Intelligence systems. Applied Soft Computing. 143. 2023.
DOI }}\url{https://doi.org/10.1016/j.asoc.2023.110421}

\foreignlanguage{english}{\textcolor{black}{NAZIR, R.; BUCAIONI, A.; PELLICCIONE, P.; Architecting ML-enabled systems:
Challenges, best practices, and design decisions. The Journal of Systems \& Software. 207. 2024. DOI
}}\url{https://doi.org/10.1016/j.jss.2023.111860}\foreignlanguage{english}{\textcolor{black}{ }}

\foreignlanguage{english}{\textcolor{black}{SERBAN, A.; BLOM, K.; HOOS, H.; VISSER, J. Software engineering practices
for machine learning – Adoption, effects, and team assessment. The Journal of Systems \& Software. 209. 2024. DOI
}}\url{https://doi.org/10.1016/j.jss.2023.111907}\foreignlanguage{english}{\textcolor{black}{ }}

\foreignlanguage{english}{\textcolor{black}{STEIDL, M.; FELDERER, M.; RAMLER, R. The pipeline for continuous development
of artificial intelligence models – Current state of research and practice. The Journal of Systems \& Software. 199.
2023. DOI }}\url{https://doi.org/10.1016/j.jss.2023.111615}\foreignlanguage{english}{\textcolor{black}{ }}

\foreignlanguage{english}{\textcolor{black}{XIN, D.; WU, E. Y.; LEE, D. J.; SALEHI, N.; PARAMESWARAN, A. Whither AutoML?
Understanding the Role of Automation in Machine Learning Workflows. In CHI Conference on Human Factors in Computing
Systems (CHI'21), Maio 8-13, 2021, Yokohama, Japão. }}\textcolor{black}{DOI
}\url{https://doi.org/10.1145/3411764.3445306}\textcolor{black}{ }



\subsection*{\textbf{2 Orientações adicionais}}



Escolha o artigo que for mais interessante para você. Utilize tradutores e o Chat GPT para entender o conteúdo dos
artigos – caso precise, mas escreva o resumo em língua portuguesa e nas suas palavras. 



Não esqueça de preencher, no trabalho, os campos relativos ao seu nome e ao artigo escolhido.



No \textit{template}, você deverá responder às seguintes questões:

\begin{itemize}
\item Qual o objetivo do estudo descrito pelo artigo?
\item Qual o problema/oportunidade/situação que levou a necessidade de realização deste estudo?
\item Qual a metodologia que os autores usaram para obter e analisar as informações do estudo?
\item Quais os principais resultados obtidos pelo estudo?
\end{itemize}


Responda cada questão utilizando o espaço fornecido no \textit{template}, sem alteração do tamanho da fonte (Times New
Roman, 10), nem alteração do espaçamento entre linhas (1.0).



Não altere as questões do template.



Utilize o editor de textos de sua preferência para preencher as respostas, mas entregue o trabalho em PDF.


%%%%%%%%%%%%%%%%%%%%%%%%%%%%%%%%%%%%%%%%%%%%%%%%%%%%%%%%%%%%%%%%%%%%%%%%%%%%%%%%%%%%%%%%%%%%%
\section*{\textbf{B - RESOLUÇÃO}}
\subsection*{\textbf{Nome do artigo escolhido:}}
\noindent Architecting ML-enabled systems: Challenges, best practices, and design decisions

\subsection*{\textbf{Qual o objetivo do estudo descrito pelo artigo?}}
O objetivo do estudo é identificar os
principais desafios, melhores práticas e
decisões de design na arquitetura de sistemas
de machine learning (ML). Os autores
buscam fornecer uma visão completa desses
aspectos para ajudar tanto pesquisadores
quanto profissionais a tomarem melhores
decisões ao projetar esses sistemas
complexos.

\subsection*{\textbf{Qual o problema/oportunidade/situação que levou à necessidade de realização desse estudo?}}
Com o aumento do uso de ML em diversas
áreas, desde recomendações até sistema
autônomos, a complexidade no
desenvolvimento e manutenção de sistemas
de ML se tornou evidente. Há uma
necessidade de compreender melhor como os
arquitetos e desenvolvedores estão
abordando essas dificuldades, especialmente
em relação a problemas de qualidade,
integração e manutenção dos modelos de ML
ao longo do tempo.

\subsection*{\textbf{Qual a metodologia que os autores usaram para obter e analisar as informações do estudo?}}
Os autores utilizaram uma metodologia
mista que incluiu uma revisão sistemática da
literatura e entrevistas com 12 especialistas
em ML de 9 países diferentes. A revisão
sistemática envolveu a análise de 3038
estudos inicialmente, que foi filtrada até um
conjunto de 41 estudos relevantes. Os dados
das entrevistas e da revisão foram analisados
tanto qualitativamente quanto
quantitativamente para encontrar padrões e
correlações entre os desafios, as melhores
práticas e as decisões de design.

\subsection*{\textbf{Quais os principais resultados obtidos pelo estudo?}}
O estudo identificou 35 desafios principais
na arquitetura de sistemas de ML, 42
melhores práticas e 27 decisões de design.
Entre os desafiso mais citados estão
questções de dados, arquitetura, e a
necessidade de evolução contínua dos
modelos. As melhores práticas incluem o uso
de arquiteturas de microserviços,
metodologias de desenvolvimento
específicas para ML e práticas de qualidade
para garantir a confiabilidade e a segurança
dos sistemas. Além disso, foram feitas
recomendações sobre como melhor gerenciar
decisões arquitêtonicas para enfrentar as
complexidades específicas dos sistemas
habilitados por ML