\label{ap:ap11}
\chapter{Visão computacional}
\section*{\textbf{A - ENUNCIADO}}

\subsection{Extração de Características }

\textcolor{black}{Os bancos de imagens fornecidos são conjuntos de imagens de 250x250 pixels de imuno-histoquímica
(biópsia) de câncer de mama. No total são 4 classes (0, 1+, 2+ e 3+) que estão divididas em diretórios.  O objetivo é
classificar as imagens nas categorias correspondentes. Uma base de imagens será utilizada para o treinamento e outra
para o teste do treino.}\textcolor{black}{ }

\textcolor{black}{As imagens fornecidas são recortes de uma imagem maior do tipo WSI }\textit{\textcolor{black}{(Whole
Slide Imaging}}\textcolor{black}{) disponibilizada pela Universidade de Warwick
(}\underline{\href{https://pubmed.ncbi.nlm.nih.gov/28771788/}{\textcolor{black}{link}}}\textcolor{black}{)}\textit{\textcolor{black}{.}}\textcolor{black}{ A nomenclatura das imagens
segue o padrão XX\_HER\_YYYY.png, onde XX é o número do paciente e YYYY é o número da imagem recortada. Separe a base
de treino em 80\% para treino e 20\% para validação. }\textbf{\textcolor{black}{Separe por pacientes (XX), não utilize
a separação randômica! Pois, imagens do mesmo paciente não podem estar na base de treino e de validação, pois isso pode
gerar um viés.}}\textcolor{black}{ No caso da CNN VGG16 remova a última camada de classificação e armazene os valores
da penúltima camada como um vetor de características. Após o treinamento, os modelos treinados devem ser validados na
base de teste.}\textcolor{black}{ }

\textcolor{black}{ }

%\textcolor{black}{Tarefas:}\textcolor{black}{ }
\subsubsection*{Tarefas:}

\begin{enumerate}[series=listWWNumxxiv,label=\alph*),ref=\alph*]
\item \textcolor{black}{Carregue a base de dados de }\textbf{\textcolor{black}{Treino.}}\textcolor{black}{ }
\item \textcolor{black}{Crie partições contendo 80\% para treino e 20\% para validação (atenção aos
pacientes).}\textcolor{black}{ }
\item \textcolor{black}{Extraia características utilizando LBP e a CNN VGG16 (gerando um csv para cada
extrator).}\textcolor{black}{ }
\item \textcolor{black}{Treine modelos Random Forest, SVM e RNA para predição dos dados extraídos.}\textcolor{black}{ }
\item \textcolor{black}{Carregue a base de }\textbf{\textcolor{black}{Teste }}\textcolor{black}{ e execute a tarefa 3
nesta base.}\textcolor{black}{ }
\item \textcolor{black}{Aplique os modelos treinados nos dados de treino}\textcolor{black}{ }
\item \textcolor{black}{Calcule as métricas de Sensibilidade, Especificidade e F1-Score com base em suas matrizes de
confusão.}\textcolor{black}{ }
\item \textcolor{black}{Indique qual modelo dá o melhor o resultado e a métrica utilizada}\textcolor{black}{ }
\end{enumerate}
\textcolor{black}{ }

\subsection{Redes Neurais}

\textcolor{black}{Utilize as duas bases do exercício anterior para treinar as Redes Neurais Convolucionais VGG16 e a
Resnet50. Utilize os pesos pré-treinados (}\textit{\textcolor{black}{Transfer Learning}}\textcolor{black}{), refaça as
camadas }\textit{\textcolor{black}{Fully Connected}}\textcolor{black}{ para o problema de 4 classes. Compare os treinos
de 15 épocas com e sem }\textit{\textcolor{black}{Data Augmentation}}\textcolor{black}{. Tanto a VGG16 quanto a
Resnet50 têm como camada de entrada uma imagem 224x224x3, ou seja, uma imagem de 224x224 pixels coloridos (3 canais de
cores). Portanto, será necessário fazer uma transformação de 250x250x3 para 224x224x3. Ao fazer o
}\textit{\textcolor{black}{Data Augmentation }}\textbf{\textit{\textcolor{black}{
}}}\textbf{\textcolor{black}{ cuidado }}\textcolor{black}{ para não alterar demais as cores das imagens e atrapalhar na
classificação.}\textcolor{black}{ }

\textcolor{black}{ }

%\textcolor{black}{Tarefas:}\textcolor{black}{ }
\subsubsection*{Tarefas:}

\begin{enumerate}[series=listWWNumxxv,label=\alph*),ref=\alph*]
\item \textcolor{black}{Utilize a base de dados de }\textbf{\textcolor{black}{Treino }}\textcolor{black}{ já separadas em
treino e validação do exercício anterior}\textcolor{black}{ }
\item \textcolor{black}{Treine modelos VGG16 e Resnet50 adaptadas com e sem }\textit{\textcolor{black}{Data
Augmentation}}\textcolor{black}{ }
\item \textcolor{black}{Aplique os modelos treinados nas imagens da base de
}\textbf{\textcolor{black}{Teste}}\textcolor{black}{ }
\item \textcolor{black}{Calcule as métricas de Sensibilidade, Especificidade e F1-Score com base em suas matrizes de
confusão.}\textcolor{black}{ }
\item \textcolor{black}{Indique qual modelo dá o melhor o resultado e a métrica utilizada}\textcolor{black}{ }
\end{enumerate}
\textcolor{black}{ }

%%%%%%%%%%%%%%%%%%%%%%%%%%%%%%%%%%%%%%%%%%%%%%%%%%%%%%%%%%%%%%%%%%%%%%%%%%%%%%%%%%%%%%%%%%%%%
\section*{\textbf{B - RESOLUÇÃO}}
\lipsum[30]