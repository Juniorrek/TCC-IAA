\label{ap:ap05}
\chapter{Estatística Aplicada II}
\section*{\textbf{A - ENUNCIADO}}

\textbf{Regressões Ridge, Lasso e ElasticNet}


\bigskip

\textbf{(100 pontos) }Fazer as regressões Ridge, Lasso e ElasticNet com a variável dependente “lwage” (salário-hora da
esposa em logaritmo neperiano) e todas as demais variáveis da base de dados são variáveis explicativas (todas essas
variáveis tentam explicar o salário-hora da esposa). No pdf você deve colocar a rotina utilizada, mostrar em uma tabela
as estatísticas dos modelos (RMSE e R\textsuperscript{2}) e concluir qual o melhor modelo entre os três, e mostrar o
resultado da predição com intervalos de confiança para os seguintes valores:

husage = 40 \ \ \ \ \ \ \ \ \ (anos – idade do marido)

husunion = 0 \ \ \ \ \ \ \ (marido não possui união estável)

husearns = 600 \ \ \ (US\$ renda do marido por semana)

huseduc = 13 \ \ \ \ \ \ (anos de estudo do marido)

husblck = 1 \ \ \ \ \ \ \ \ \ \ (o marido é preto)

hushisp = 0 \ \ \ \ \ \ \ \ \ \ (o marido não é hispânico)

hushrs = 40 \ \ \ \ \ \ \ \ \ \ (horas semanais de trabalho do marido)

kidge6 = 1 \ \ \ \ \ \ \ \ \ \ \ \ (possui filhos maiores de 6 anos)

age = 38 \ \ \ \ \ \ \ \ \ \ \ \ \ \ \ (anos – idade da esposa)

black = 0 \ \ \ \ \ \ \ \ \ \ \ \ \ \ (a esposa não é preta)

educ = 13 \ \ \ \ \ \ \ \ \ \ \ \ \ (anos de estudo da esposa)

hispanic = 1 \ \ \ \ \ \ \ \ \ (a esposa é hispânica)

union = 0 \ \ \ \ \ \ \ \ \ \ \ \ \ (esposa não possui união estável)

exper = 18 \ \ \ \ \ \ \ \ \ \ \ (anos de experiência de trabalho da esposa)

kidlt6 = 1 \ \ \ \ \ \ \ \ \ \ \ \ \ (possui filhos menores de 6 anos)


\bigskip

obs: lembre-se de que a variável dependente “lwage” já está em logarítmo, portanto voçê não precisa aplicar o logaritmo
nela para fazer as regressões, mas é necessário aplicar o antilog para obter o resultado da predição. 


%%%%%%%%%%%%%%%%%%%%%%%%%%%%%%%%%%%%%%%%%%%%%%%%%%%%%%%%%%%%%%%%%%%%%%%%%%%%%%%%%%%%%%%%%%%%%
\section*{\textbf{B - RESOLUÇÃO}}
\lipsum[30]