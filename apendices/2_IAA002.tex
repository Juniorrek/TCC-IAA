\label{ap:ap02}
\chapter{Linguagem de Programação Aplicada}

\section*{\textbf{A - ENUNCIADO}}
\textbf{Nome da base de dados do exercício}: \textit{precos\_carros\_brasil.csv}

\textbf{Informações sobre a base de dados: }

Dados dos preços médios dos carros brasileiros, das mais diversas marcas, no ano de 2021, de acordo com dados extraídos
da tabela FIPE (Fundação Instituto de Pesquisas Econômicas). A base original foi extraída do site Kaggle
(\href{https://www.kaggle.com/datasets/vagnerbessa/average-car-prices-bazil/data}{\textcolor[HTML]{1155CC}{Acesse aqui
a base original}}). A mesma foi adaptada para ser utilizada no presente exercício.

Observação: As variáveis \textit{fuel\hspace{0pt}\hspace{0pt}}, \textit{gear} e \textit{engine\_size} foram extraídas
dos valores da coluna \textit{model}, pois na base de dados original não há coluna dedicada a esses valores. Como
alguns valores do modelo não contêm as informações do tamanho do motor, este conjunto de dados não contém todos os
dados originais da tabela FIPE.

\textbf{Metadados:}
\begin{center}
\begin{tabular}{|p{0.4\textwidth}|p{0.55\textwidth}|}
\hline
\textbf{Nome do campo} & \textbf{Descrição} \\
\hline
year\_of\_reference & O preço médio corresponde a um mês do ano de referência \\
\hline
month\_of\_reference & O preço médio corresponde a um mês específico, pois a FIPE atualiza mensalmente \\
\hline
fipe\_code & Código único da FIPE \\
\hline
authentication & Código único de autenticação para consulta FIPE \\
\hline
brand & Marca do carro \\
\hline
model & Modelo do carro \\
\hline
fuel & Tipo de combustível \\
\hline
gear & Tipo de engrenagem \\
\hline
engine\_size & Tamanho do motor em centímetros cúbicos \\
\hline
year\_model & Ano do modelo (pode ser diferente do ano de fabricação) \\
\hline
avg\_price & Preço médio do carro em reais \\
\hline
\end{tabular}
\end{center}



\textbf{\textit{Atenção: ao fazer o download da base de dados, selecione o formato .csv. É o formato que será
considerado correto na resolução do exercício.}}



\newpage
\subsection{Análise Exploratória dos dados}
%\textbf{1 Análise Exploratória dos dados}



A partir da base de dados \textbf{precos\_carros\_brasil.csv}, execute as seguintes tarefas:

\begin{enumerate}[series=listWWNumxi,label=\alph*.,ref=\alph*]
\item Carregue a base de dados media\_precos\_carros\_brasil.csv
\item Verifique se há valores faltantes nos dados. Caso haja, escolha uma tratativa para resolver o problema de valores
faltantes
\item Verifique se há dados duplicados nos dados
\item Crie duas categorias, para separar colunas numéricas e categóricas. Imprima o resumo de informações das variáveis
numéricas e categóricas (estatística descritiva dos dados)
\item Imprima a contagem de valores por modelo (model) e marca do carro (brand)
\item Dê um breve explicação (máximo de quatro linhas) sobre os principais resultados encontrados na Análise
Exploratória dos dados
\end{enumerate}


\subsection{Visualização dos dados}
%\textbf{2 Visualização dos dados}

A partir da base de dados \textbf{precos\_carros\_brasil.csv,} execute as seguintes tarefas:

\begin{enumerate}[series=listWWNumx,label=\alph*.,ref=\alph*]
\item Gere um gráfico da distribuição da quantidade de carros por marca
\item Gere um gráfico da distribuição da quantidade de carros por tipo de engrenagem do carro
\item Gere um gráfico da evolução da média de preço dos carros ao longo dos meses de 2022 (variável de tempo no eixo X)
\item Gere um gráfico da distribuição da média de preço dos carros por marca e tipo de engrenagem
\item Dê uma breve explicação (máximo de quatro linhas) sobre os resultados gerados no item d
\item Gere um gráfico da distribuição da média de preço dos carros por marca e tipo de combustível
\item Dê uma breve explicação (máximo de quatro linhas) sobre os resultados gerados no item f
\end{enumerate}






\subsection{Aplicação de modelos de machine learning para prever o preço médio dos carros}
%\textbf{3 Aplicação de modelos de machine learning para prever o preço médio dos carros}




{\centering
A partir da base de dados \textbf{precos\_carros\_brasil.csv}, execute as seguintes tarefas:
\par}

\begin{enumerate}[series=listWWNumxii,label=\alph*.,ref=\alph*]
\item Escolha as variáveis \textbf{numéricas} (modelos de Regressão) para serem as variáveis independentes do modelo.A
variável target é \textbf{avg\_price. Observação:} caso julgue necessário, faça a transformação de variáveis
categóricas em variáveis numéricas para inputar no modelo. Indique \textbf{quais variáveis} foram transformadas e
\textbf{como} foram transformadas
\item Crie partições contendo 75\% dos dados para treino e 25\% para teste
\item Treine modelos RandomForest (biblioteca RandomForestRegressor) e XGBoost (biblioteca XGBRegressor) para predição
dos preços dos carros. \textbf{Observação}: caso julgue necessário, mude os parâmetros dos modelos e rode novos
modelos. Indique quais parâmetros foram inputados e indique o treinamento de cada modelo
\item Grave os valores preditos em variáveis criadas
\item Realize a análise de importância das variáveis para estimar a variável target, \textbf{para cada modelo treinado}
\item Dê uma breve explicação (máximo de quatro linhas) sobre os resultados encontrados na análise de importância de
variáveis
\item Escolha o melhor modelo com base nas métricas de avaliação MSE, MAE e R²
\item Dê uma breve explicação (máximo de quatro linhas) sobre qual modelo gerou o melhor resultado e a métrica de
avaliação utilizada
\end{enumerate}

\section*{\textbf{B - RESOLUÇÃO}}
\lipsum[30]