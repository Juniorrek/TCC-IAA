\chapter{Parecer Técnico} \label{cha:parecertecnico}

%%% Intro IA 
A inteligência sempre foi uma característica fundamental do ser humano. Desde os primórdios, buscamos compreender como o nosso cérebro é capaz de perceber, aprender, raciocinar e agir em um mundo complexo. O campo da Inteligência Artificial (IA) surge nessa mesma vertente. Entretanto, segundo \textcite{russell2022}, essa área de estudo vai além e procura não apenas entender, mas também construir entidades inteligentes, capazes de realizar tarefas que normalmente exigem a inteligência humana.

O campo da IA surgiu oficialmente em 1956, derivando subáreas como reconhecimento de padrões, tomada de decisão, raciocínio lógico e aprendizado a partir de dados. Desde então evoluiu de abordagens baseadas em regras fixas para métodos mais sofisticados de aprendizado de máquina e aprendizado profundo (Deep Learning), possibilitando o surgimento de aplicações em diferentes setores, como saúde, indústria, entretenimento e comunicação \cite{nilsson2010quest}.

%%CITANDO SUBAREAS 2X, COLOCAR SÓ EM 1 PARAGRAFO, NEM QUE SEPARE GRANDES AREAS E AREAS ESPECIFICAS...

%%% Subareas IA
Cada diferente subárea é  voltada a um tipo específico de desafio. Entre elas, pode-se destacar a visão computacional, que permite o reconhecimento e interpretação de imagens; o reconhecimento de fala, que transforma áudio em texto; os sistemas especialistas, que simulam a tomada de decisão em domínios específicos; e a robótica inteligente, que combina sensores, planejamento e ação em ambientes físicos.

%%% NLP IA

Dentro desse conjunto, destaca-se o Processamento de Linguagem Natural (NLP - Natural Language Processing), responsável por ensinar máquinas a compreender, interpretar e produzir linguagem humana. Essa tarefa é desafiadora, pois a linguagem é complexa, ambígua e profundamente dependente de contexto cultural e situacional. Computadores conseguem interpretar facilmente linguagens estruturadas, como linguagens de programação, mas não entendem diretamente a linguagem natural, que exige interpretação e contexto \cite{russell2022}.

%%% Exemplo NLP 
Por exemplo, um navegador pode executar corretamente um código em JavaScript que define que, ao rolar a página, seja exibido um alerta na tela:
\begin{lstlisting}[language=Python, style=input]
# Código que o computador entende
temperatura = 30

if temperatura > 25:
    print("A temperatura está alta!")
\end{lstlisting}

\noindent Por outro lado, uma instrução equivalente em linguagem natural:
\begin{tcolorbox}[
  colback=yellow!20, 
  colframe=black,
  width=0.8\linewidth,   % largura do box igual ao lstlisting
  left=0pt,               % remove deslocamento interno
  boxsep=2mm,
  enlarge left by=0.1\linewidth  % desloca o box inteiro para direita
]
Se a temperatura estiver acima de 25 graus, avise que a temperatura está alta.
\end{tcolorbox}
Tal instrução não pode ser interpretada diretamente pela máquina, justamente porque o texto humano é ambíguo e depende de contexto. 
O NLP surge para reduzir essa lacuna, permitindo que computadores sejam capazes de interpretar comandos e informações expressos em linguagem natural, ampliando a interação entre humanos e sistemas computacionais.

%%% Linguistica NLP

Um dos maiores desafios atacados pela NLP é a ambiguidade linguística. A linguagem humana raramente possui um único significado literal; palavras e frases podem assumir sentidos diferentes dependendo do contexto. Por exemplo, a sentença em inglês “I saw the man with the telescope”, quem tem o telescópio, o sujeito da oração ou o homem? Essa multiplicidade de interpretações demonstra como a compreensão automática requer não apenas reconhecer palavras, mas também processá-las com base em contexto sintático, semântico e pragmático. Portanto, o NLP não se resume a um simples processamento de texto, mas a um esforço para modelar matematicamente a própria complexidade da linguagem \cite{jurafsky2023speech}.

%%% Tradução idiomas NLP

Outra aplicação da NLP é a tradução entre linguas, cuja evolução demonstra claramente a importância do contexto linguístico. Segundo \textcite{hutchins2005history}, os primeiros tradutores automáticos, desenvolvidos nas décadas de 1950 e 1960, utilizavam abordagens baseadas em regras gramaticais rígidas, resultando em traduções literais que muitas vezes ignoravam expressões idiomáticas e nuances culturais.

Com o avanço dos métodos estatísticos e, mais recentemente, das redes neurais, as traduções passaram a considerar padrões contextuais linguísticos aprendidos a partir de grandes volumes de dados. Em vez de traduzir palavra por palavra, os sistemas modernos analisam frases inteiras, levando em conta o significado geral e a probabilidade de combinações linguísticas. Essa abordagem, conhecida como tradução neural, revolucionou ferramentas amplamente utilizadas, como o Google Translate e o DeepL, tornando a tradução automática muito mais natural e próxima da interpretação humana.

%%% LLMS NLP

Outra aplicação da NLP popularmente conhecida ultimamente são os chamados Modelos de Linguagem de Grande Escala (LLMs - Large Language Models), como o ChatGPT. Tais modelos são capazes de compreender e gerar texto de forma coerente, criando respostas contextualizadas, resumos, traduções e até textos criativos. O funcionamento das LLMs baseia-se em arquiteturas de redes neurais profundas, especialmente os \textit{transformers}, introduzidos por \textcite{\cite{vaswani2017attention}, que revolucionaram o campo ao permitir o processamento paralelo de sequências de palavras e a captura de relações de longo alcance no texto.

O impacto dos LLMs transcende o campo técnico: eles transformaram a forma como interagimos com a tecnologia, permitindo a comunicação natural entre humanos e máquinas. Hoje, sistemas baseados em NLP são empregados em assistentes virtuais, ferramentas de revisão gramatical, análise de sentimentos, geração de conteúdo e ensino de idiomas, demonstrando o potencial da IA em compreender e reproduzir aspectos cada vez mais complexos da linguagem humana.


\begin{venndiagram3sets}[labelOnlyA=IA,
    labelOnlyB=Machine Learning,
    labelOnlyC=Data Science,
    labelABC={Deep Learning, NLP, LLMs}]
\end{venndiagram3sets}