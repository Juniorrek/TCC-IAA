\label{ap:ap07}
\chapter{Aspectos filosóficos e éticos da IA}
\section*{\textbf{A - ENUNCIADO}}
%%%%%%%%%%%NO MODELO SO VEM DPS DE VISÃO COMPUTACIONAL

\textbf{Título do Trabalho:} {\textquotedbl}Estudo de Caso: Implicações Éticas do Uso do ChatGPT{\textquotedbl}



\textbf{Trabalho em Grupo:} O trabalho deverá ser realizado em grupo de alunos de no máximo seis (06) integrantes.



\textbf{Objetivo do Trabalho:} Investigar as implicações éticas do uso do ChatGPT em diferentes contextos e propor soluções
responsáveis para lidar com esses dilemas.

\textbf{Parâmetros para elaboração do Trabalho:}



\textbf{1. Relevância Ética}: O trabalho deve abordar questões éticas significativas relacionadas ao uso da inteligência
artificial, especialmente no contexto do ChatGPT. Os alunos devem identificar dilemas éticos relevantes e explorar como
esses dilemas afetam diferentes partes interessadas, como usuários, desenvolvedores e a sociedade em geral.

\textbf{2. Análise Crítica}: Os alunos devem realizar uma análise crítica das implicações éticas do uso do ChatGPT em
estudos de caso específicos. Eles devem examinar como o algoritmo pode influenciar a disseminação de informações, a
privacidade dos usuários e a tomada de decisões éticas. Além disso, devem considerar possíveis vieses algorítmicos,
discriminação e questões de responsabilidade.

\textbf{3. Soluções Responsáveis}: Além de identificar os desafios éticos, os alunos devem propor soluções responsáveis
e éticas para lidar com esses dilemas. Isso pode incluir sugestões para políticas, regulamentações ou práticas de
design que promovam o uso responsável da inteligência artificial. Eles devem considerar como essas soluções podem
equilibrar os interesses de diferentes partes interessadas e promover valores éticos fundamentais, como transparência,
justiça e privacidade.

\textbf{4. Colaboração e Discussão}: O trabalho deve envolver discussões em grupo e colaboração entre os alunos. Eles
devem compartilhar ideias, debater diferentes pontos de vista e chegar a conclusões informadas através do diálogo e da
reflexão mútua. O estudo de caso do ChatGPT pode servir como um ponto de partida para essas discussões, incentivando os
alunos a aplicar conceitos éticos e legais aprendidos ao analisar um caso concreto.

\textbf{5. Limite de Palavras}: O trabalho terá um limite de 6 a 10 páginas teria aproximadamente entre 1500 e 3000
palavras.

\textbf{6. Estruturação Adequada}: O trabalho siga uma estrutura adequada, incluindo introdução, desenvolvimento e
conclusão. Cada seção deve ocupar uma parte proporcional do total de páginas, com a introdução e a conclusão ocupando
menos espaço do que o desenvolvimento.

\textbf{7. Controle de Informações}: Evitar incluir informações desnecessárias que possam aumentar o comprimento do
trabalho sem contribuir significativamente para o conteúdo. Concentre-se em informações relevantes, argumentos sólidos
e evidências importantes para apoiar sua análise.

\textbf{8. Síntese e Clareza}: O trabalho deverá ser conciso e claro em sua escrita. Evite repetições desnecessárias e
redundâncias. Sintetize suas ideias e argumentos de forma eficaz para transmitir suas mensagens de maneira sucinta. 

\textbf{9. Formatação Adequada}: O trabalho deverá ser apresentado nas normas da ABNT de acordo com as diretrizes
fornecidas, incluindo margens, espaçamento, tamanho da fonte e estilo de citação. Deve-se seguir o seguinte template de
arquivo: \url{https://bibliotecas.ufpr.br/wp-content/uploads/2022/03/template-artigo-de-periodico.docx}


%%%%%%%%%%%%%%%%%%%%%%%%%%%%%%%%%%%%%%%%%%%%%%%%%%%%%%%%%%%%%%%%%%%%%%%%%%%%%%%%%%%%%%%%%%%%%
\section*{\textbf{B - RESOLUÇÃO}}
\subsection*{\textbf{Estudo de Caso: Implicações Éticas do Uso do ChatGPT}}
%\noindent Caroline Pereira de Sena\\
%David Reksidler Junior \\
%Hideraldo Luis Simon Junior\\
%Marco Antonio Pedroso Vicente

\textbf{RESUMO}

O rápido avanço da inteligência artificial (IA) levou ao desenvolvimento de modelos de linguagem sofisticados como o ChatGPT. Este trabalho explora as implicações éticas do uso de tal ferramenta, focando em seu impacto em áreas como educação, ambiente de trabalho e meio artístico. Ao examinar tais estudos de caso específicos, este trabalho identificou alguns dilemas éticos significativos e analisou seus efeitos sobre diferentes partes interessadas. Além disso, o presente trabalho propõe algumas possíveis soluções responsáveis para que esses dilemas sejam superados, promovendo um uso mais positivo para a sociedade em geral. O objetivo é fornecer uma compreensão abrangente do panorama ético em torno do ChatGPT e sugerir políticas que garantam seu uso benéfico e equitativo na sociedade.

Palavras-chave: Ética. ChatGPT. Inteligência artificial. Análise crítica. Soluções responsáveis.


\textbf{ABSTRACT}

The rapid advance of artificial intelligence (AI) has led to the development of sophisticated language models such as ChatGPT. This paper explores the ethical implications of using such a tool, focusing on its impact in areas such as education, the workplace and the arts. By examining such specific case studies, this paper has identified some significant ethical dilemmas and analyzed their effects on different stakeholders. In addition, this paper proposes some possible responsible solutions so that these dilemmas can be overcome, promoting a more positive use for society in general. The aim is to provide a comprehensive understanding of the ethical landscape surrounding ChatGPT and to suggest policies that ensure its beneficial and equitable use in society.


Keywords: Ethics. ChatGPT. Artificial intelligence. Critical analysis. Responsible solutions.


\subsection*{\textbf{1 INTRODUÇÃO}}
A inteligência artificial (IA) tem evoluído rapidamente e ganhado muita popularidade recentemente. Hoje em dia, falar sobre IA é um assunto muito comum, diferente de alguns poucos anos atrás, e grande parte das pessoas acaba por utilizar serviços que utilizam inteligência artificial de alguma forma ou de outra mesmo sem saber.

Esse rápido avanço abre espaço para discussões éticas e filosóficas de como essa tecnologia pode e poderá afetar a vida das pessoas no futuro. Uma das aplicações recentes mais notáveis é o ChatGPT, uma tecnologia de IA capaz de gerar textos e realizar conversas de maneira natural. 

Este trabalho tem como objetivo explorar algumas implicações éticas do uso do ChatGPT em diversos cenários, tais como na educação, no ambiente de trabalho e também no meio artístico. Além de explorar tais implicações também investigar alguns dos dilemas que a inteligência artificial em geral apresenta hoje e por fim propor algumas soluções responsáveis para os problemas identificados.

A análise será estruturada em duas seções principais: análise crítica, onde abordaremos estudos de casos específicos para ilustrar os impactos éticos da tecnologia; e soluções responsáveis, onde serão propostas medidas para mitigar os desafios identificados e promover o uso junto e transparente do ChatGPT em tais estudos de caso. Ao realizar essa análise, esperamos contribuir para um entendimento mais profundo das implicações éticas da inteligência artificial e para o desenvolvimento de boas práticas que garantam o seu uso benéfico na sociedade, minimizando os potenciais usos negativos.

\subsection*{{1.1 CHATGPT}}
O ChatGPT é um modelo de inteligência artificial que pode entender e gerar textos como se fosse uma pessoal real. Ele foi criado para conversar com as pessoas, responder perguntas, fornecer informações e até mesmo ajudar em algumas tarefas. A tecnologia de inteligência artificial permite que o ChatGPT aprenda com diversos exemplos de conversas e textos, o que o ajuda a fornecer respostas úteis e em grande parte também corretas.

De maneira mais técnica, o ChatGPT é um modelo de linguagem treinando para gerar texto e foi otimizado para diálogos, utilizando aprendizado por reforço, com feedback humano, um método que utiliza demonstrações humanas e comparações de preferências para orientar o modelo em direção ao comportamento desejado \cite{OpenAI}.



\subsection*{\textbf{2 ANÁLISE CRÍTICA}}
\subsection*{{2.1 ESTUDO DE CASO 1: USO NA EDUCAÇÃO}}
No contexto educacional o ChatGPT pode ser uma ferramenta valiosa para auxiliar professores no ensino dos alunos. Tal ferramenta é capaz de gerar respostas de forma rápida e diferentes, ajudando na resolução de problemas. Entretanto, há preocupações em relação ao seu uso, tanto para os corpo docente como para o discente, possibilitando a geração de um determinado nível de dependência dessa tecnologia, podendo comprometer suas capacidades intelectuais, pensamento crítico e criatividade, sem o uso da ferramenta. O processo de transformação, principalmente o digital, exige dos profissionais envolvidos com a educação, um olhar diferenciado nos planejamentos e suas respectivas aplicações, pois há muitos que são analfabetos funcionais em conduzir determinadas metodologias no processo ensino-aprendizagem, um exemplo, são as metodologias ativas, em que há uma interpretação equivocada entre gamificação e jogos. 

Nesse processo, o papel do professor não será como o de outrora em que tal personagem era a única fonte de conhecimento. Hoje e no futuro se desenham cenários em que a docência exercerá muito mais o papel de mediação pedagógica entre os objetos de conhecimento e os estudantes. Por isso, o educar pela pesquisa e extensão se tornam tão significativos \cite{cassol2023}.

Para que se tenha uma qualidade na construção do saber, a entrada das informações, que devem ser interativas e desafiadoras para o aluno em  desenvolvimento para se tornar um estudante, o professor deverá estar em prontidão, não utilizando a ferramenta como “bengala” para conduzir suas aulas, bem como oferecer um suporte com maior qualidade e orientações possibilitando uma diversidade de caminhos para explorar o mundo dos saberes com consciência. Em relação aos discentes, respeitando o seu tempo de maturação na forma de perceber o mundo, fomentá-los por meio da disciplina em realizar as tarefas propostas, iniciando com uma boa orientação, sobre como aplicar a ferramenta nas determinadas áreas, utilizando a transdisciplinaridade e autoconhecimento para a ampliação da base de informações a serem utilizados na construção do pensamento crítico, sendo mais assertivo nas perguntas à serem realizadas, aumentado a probabilidade de obter melhores respostas, e na sua interpretação e execução, obter um resultado mais original, personalizado; não necessitando se utilizar do plágio ou qualquer outro artifício que não contribua com a melhoria contínua do ChatGPT, no contexto escolar e outras áreas. 


\subsection*{{2.2 ESTUDO DE CASO 2: USO NO AMBIENTE DE TRABALHO}}
No ambiente de trabalho, o uso de inteligências artificiais promete trazer um grande aumento da produtividade e precisão, além da simplificação de tarefas, o que permitiria liberar tempo para tarefas que exigem mais criatividade e pensamento crítico. O ChatGPT, nesse contexto, vem sendo visto como uma ótima ferramenta para análise de dados, organização de informações, automatização de tarefas repetitivas e geração de código, e já é amplamente popular em diversas áreas, principalmente dentro do mundo corporativo. Segundo \textcite{lopes2024}, 89\% dos profissionais de TI e 86\% do setor de marketing, já adotaram o uso do ChatGPT numa frequência, ao menos, mensal.

Apesar dos benefícios da introdução do ChatGPT no ambiente de trabalho, surgem, a partir disso, uma série de questionamentos e preocupações a respeito das mudanças que essa tecnologia pode causar. 

Uma das áreas a serem afetadas seria o mercado de trabalho, já que um dos principais questionamentos é se a ferramenta poderia ser capaz de causar a extinção de algumas carreiras. O ChatGPT parece ter o potencial de substituir pessoas na realização de algumas tarefas, como por exemplo: revisão de contratos, resumo de textos, criação de apresentações, desenvolvimento de código, organização de cadeias de produção; afetando assim profissões como direito, jornalismo, engenharia de software e magistério \cite{sutto2023}. Apenas a remota possibilidade de que isso aconteça, poderia ser a causa do aumento do nível de ansiedade de muitos trabalhadores, talvez, gerando um problema de saúde pública. Contudo, ao mesmo tempo, há a expectativa de que surjam novas áreas de trabalho,  para profissionais especializados no uso da ferramenta, que sejam capazes de integrá-la em diferentes setores.

Além do mercado de trabalho, é válido buscar entender o efeito que o uso de ferramentas como o ChatGPT teria sobre o perfil dos trabalhadores. É importante cogitar que a intensificação do uso do ChatGPT poderá afetar as habilidades dos trabalhadores, no sentido de que ao depender muito da ferramenta como uma “muleta”, os profissionais acabem “atrofiando” suas habilidades. Poderia haver diminuição na habilidade de comunicação escrita, pois o uso da IA para gerar conteúdos textuais em alta frequência reduz o não aprimoramento e o esquecimento de técnicas de escrita. Por sua vez, a redução da habilidade de comunicação escrita também tem impacto na comunicação verbal, pois afeta a habilidade de formulação de argumentos e organização de idéias, e então pode ser a causa de mais problemas de comunicação no ambiente de trabalho. 

A inibição da criatividade também poderia ser uma outra consequência, visto que o uso do ChatGPT ajuda na otimização de produção de texto inédito porém não gera conhecimento novo, considerando que é treinado com conhecimento produzido no passado. A tendência é que o conteúdo produzido pelo uso recorrente do ChatGPT gere uma repetição ou reprodução de informações entre os profissionais, em variados setores e sem avanço ou melhoria. Em nome da otimização do tempo, a inovação é inibida e se diminui o hábito e o exercício da criatividade.

Surge também a dúvida sobre até que ponto uma inteligência artificial pode ser utilizada em tarefas que dependem de pesquisa e consulta a informações verdadeiras. Com uma compreensão mais aprofundada do funcionamento de inteligências artificiais, sabe-se que elas não são capazes de raciocinar, compreender o contexto e aspectos humanos relacionados, apenas são capazes de abstrair dados e são capazes de gerar dados fictícios quando não possuem as informações verdadeiras. Porém ferramentas como o ChatGPT passaram a ser usadas por muitos como uma fonte fácil de conhecimento verdadeiro, o que gera a necessidade de treinar profissionais para aprender a usar a ferramenta e analisar criticamente se o conteúdo gerado é válido ou não.

O aspecto da experiência do usuário também passa a ser afetado, devido ao aumento de ferramentas automatizadas para atendimento aos clientes. Não é incomum, ao acessar o site de uma empresa e ao se direcionar para o serviço de atendimento ao cliente, o mesmo ser redirecionado para uma conversa com robôs. Tais robôs em grande maioria se utilizam de ferramentas geradoras de textos como o ChatGPT para se comunicarem e auxiliarem na resolução de problemas comuns entre clientes de uma determinada empresa. Isso de certa forma pode melhorar a eficiência de diversos atendimentos, fornecendo respostas rápidas e quase sempre precisas. Entretanto, embora leve a um aumento na eficiência operacional do atendimento, muitas pessoas reclamam da “desumanização” nesses processos, onde os clientes geralmente sentem falta de interações humanas autênticas, principalmente se tratando de problemas mais complexos que nem sempre os robôs conseguem ajudar. 


\subsection*{{2.3 ESTUDO DE CASO 3: USO NO MEIO ARTÍSTICO}}
A inteligência artificial tem sido cada vez mais utilizada em diversas áreas criativas para a produção de novas artes, desde músicas até pinturas ou histórias. O ChatGPT em específico com sua enorme capacidade em gerar textos pode ser utilizado para criação de livros, roteiros, poesias, letras de músicas e até mesmo colaborar criativamente na criação de outros tipos de obras de arte. Para fins de simplicidade no texto chamaremos qualquer produção artística de obra. 

Apesar dos erros e falhas ainda presentes, a habilidade já demonstrada pelo chat e seu potencial de se aprimorar ainda mais no futuro vem gerando não apenas admiração, mas também alguns receios \cite{suzuki2023}. Este estudo de caso examina as implicações éticas do uso do ChatGPT no meio artístico, analisando como tal tecnologia influencia a criatividade humana, a originalidade, os direitos autorais e a autenticidade das obras de artes.

Conforme brevemente citado anteriormente, o ChatGPT pode ser utilizado no meio artístico para várias finalidades como:

\begin{itemize}
    \item Criação de histórias e roteiros: Escritores ou até mesmo cineastas podem utilizar a ferramenta para gerar diálogos, desenvolver roteiros e histórias, séries ou peças de teatro.
    \item Poesia e literatura: Poetas e escritores podem obter auxílio do ChatGPT para criar poemas, contos e romances.
    \item Composição musical: Músicos podem utilizar a ferramentas para escrever letras de música ou até mesmo compor melodias.
    \item Artes visuais: Artistas podem utilizar de um auxílio criativo do chat para gerar descrições e conceitos de obras de artes que podem ser transformados em arte visual.
\end{itemize}

A principal preocupação ética relacionada ao uso de inteligência artificial no meio artístico é a respeito da originalidade dos conteúdos gerados. A arte tradicionalmente valoriza a expressão individual e criativa que existe em cada ser humano, o uso de um modelo de inteligência artificial que é treinado a partir de obras já existentes rompe com essa tradição. Se uma obra é criada totalmente ou com certa assistência do ChatGPT, até que ponto ela pode ser considerada uma expressão genuína de um artista humano?

Partindo da preocupação anterior, além da originalidade artística, outro ponto a se considerar é o dos direitos autorais. Tais obras geradas criam enormes desafios para a determinação de autoria de propriedade intelectual. De maneira geral, se uma IA contribui para a criação de uma obra, quem deve ser creditado como autor? O desenvolvedor da IA, o usuário que a utilizou, ambos, ou os autores a qual o modelo se baseou durante o treinamento?

Além dessa preocupação com direitos autorais na criação de uma obra nova, também há a preocupação sobre a violação de direitos autorais sobre obras já existentes, tendo em vista que os modelos são treinados sobre um conjunto já existente de obras. Isso levanta a questão de possível infração de direito autoral já que o modelo pode gerar conteúdo semelhante a essas obras. E neste caso novamente fica a pergunta, quem deve ser responsabilizado? Em várias partes do mundo cresce um movimento que defende mudanças nas leis de direitos autorais para que elas passem a proteger as obras dos artistas humanos das obras geradas por inteligência artificial \cite{prado2023}. 


\subsection*{\textbf{3 SOLUÇÕES RESPONSÁVEIS}}
\subsection*{{3.1 PRIVACIDADE}}
Como toda tecnologia emergente, na Inteligência Artificial Generativa existem riscos em sua utilização. Uma das preocupações do uso generalizado desta tecnologia é o impacto na privacidade de dados. Desta forma, “para colher os benefícios dessa tecnologia inovadora, é crucial construir uma relação de confiança com o mercado e com os diversos \textit{stakeholders}, diminuindo preocupações a respeito de como os dados serão utilizados.” \cite{kpmg2024}

O movimento de órgãos reguladores é essencial para balizar o uso e definir os direitos e deveres. Por exemplo, o Parlamento Europeu publicou, em 2023, o AI Act e no Brasil estamos discutindo através do Projeto de Lei nº 2338, de 2023 (\url{https://www25.senado.leg.br/web/atividade/materias/-/materia/157233}). Com os balizadores criados diminui-se as preocupações, tanto dos usuários, quanto dos stackholders. Garantindo aos usuários o controle e a consciência de como seus dados serão utilizados e possibilitando o crescimento exponencial no uso da tecnologia. 

\subsection*{{3.2 TRANSPARÊNCIA}}
Embora alguns modelos tenham uma certa natureza imprevisível, a transparência dos processos que ocorrem durante o treinamento e utilização de modelos geradores de texto são essenciais para construir a confiança dos usuários. 

As empresas que desenvolvem esse tipo de tecnologia devem ser transparentes quanto a como seus modelos funcionam, incluindo detalhes sobre os dados utilizados para treinamento e os processos de tomada de decisão.

Utilizando como exemplo a AI Act, publicada em 2023, pelo Parlamento Europeu, temos a seguinte diretriz:

“A IA generativa, como o ChatGPT, não será classificada como de alto risco, mas terá que cumprir os requisitos de transparência e a lei de direitos autorais da UE:

\begin{itemize}
    \item Divulgar que o conteúdo foi gerado por IA.
    \item Projetar o modelo para evitar a geração de conteúdo ilegal.
    \item Publicação de resumos de dados protegidos por direitos autorais usados para treinamento \cite{ep2023}.
\end{itemize}

Para o uso de transparência nos modelos, existem projetos open sources que facilitam a implantação, como o TrustyAI (\url{https://github.com/trustyai-explainability}).

\subsection*{{3.3 VIESES}}
O viés de IA é como chamamos a ocorrência de resultados tendenciosos devido a vieses que distorcem os dados de treinamentos levando a resultados distorcidos. Na prática, os modelos absorvem preconceitos humanos, através das bases de treinamento, e os reproduzem em seus resultados.

Há uma série de vieses possíveis em um modelo de IA. Segundo o artigo “O que é viés de IA?” \cite{ibm2023} temos os seguintes tipos:

\begin{itemize}
\item Viés do algoritmo
\item Viés cognitivo
\item Viés de confirmação
\item Viés de exclusão
\item Viés de medição
\item Viés de homogeneidade fora do grupo
\item Viés de preconceito
\item Viés de recordação
\item Viés de amostragem/seleção
\item Viés de estereótipo 
\end{itemize}

Para que a IA não possua vieses é necessário analisar os dados de treinamentos e tomar medidas para a diversidade dos dados, garantindo dados completos, imparciais e sem preconceitos. 


\subsection*{\textbf{4 CONCLUSÃO}}
A inteligência artificial representa um grande avanço tecnológico para a humanidade. O uso do ChatGPT em específico trouxe inúmeros benefícios capazes de transformar diversos setores da nossa sociedade, desde ambientes educacionais até meios artísticos. No entanto, com grandes poderes vêm grandes responsabilidades, e essa ferramenta poderosa também levanta questões éticas importantes a serem discutidas.

Ao longo deste trabalho exploramos somente algumas dessas implicações éticas, e analisamos como essas implicações afetam diferentes partes interessadas, desde usuários, desenvolvedores e a sociedade em geral. Além disso, propusemos algumas soluções que podem ajudar a mitigar os problemas apresentados, promovendo um uso mais positivo, justo, transparente e ético da tecnologia.

É essencial que continuemos discutindo e desenvolvendo políticas, visando a garantia que a evolução da inteligência artificial esteja sempre alinhada com princípios éticos, tendo em vista a forte tendência de que a velocidade da evolução aumente cada vez mais. Somente assim poderemos, como sociedade, aproveitar todo o potencial que essa tecnologia tem a oferecer.