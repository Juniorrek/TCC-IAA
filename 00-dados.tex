%%%%%%%%%%%%%%%%%%%%%%%%%%%%%%%%%%%%%%%%%%%%%%%%%%%%%%%
% Arquivo para entrada de dados para a parte pré textual
%%%%%%%%%%%%%%%%%%%%%%%%%%%%%%%%%%%%%%%%%%%%%%%%%%%%%%%
% 
% Basta digitar as informações indicidas, no formato 
% apresentado.
%
%%%%%%%
% Os dados solicitados são, na ordem:
%
% tipo do trabalho
% componentes do trabalho 
% título do trabalho
% nome do autor
% local 
% data (ano com 4 dígitos)
% orientador(a)
% coorientador(a)(as)(es)
% arquivo com dados bibliográficos
% instituição
% setor
% programa de pós gradução
% curso
% preambulo
% data defesa
% CDU
% errata
% assinaturas - termo de aprovação
% resumos & palavras chave
% agradecimentos
% dedicatoria
% epígrafe


% Informações de dados para CAPA e FOLHA DE ROSTO
%----------------------------------------------------------------------------- 
\tipotrabalho%{Trabalho Acadêmico}
    {Relatório Técnico}
%    {Dissertação}
%    {Tese}
%    {Monografia}

% Marcar Sim para as partes que irão compor o documento pdf
%----------------------------------------------------------------------------- 
 \providecommand{\terCapa}{Sim}
 \providecommand{\terFolhaRosto}{Sim}
 \providecommand{\terTermoAprovacao}{Sim}
 \providecommand{\terDedicatoria}{Nao}
 \providecommand{\terFichaCatalografica}{Nao}
 \providecommand{\terEpigrafe}{Sim}
 \providecommand{\terAgradecimentos}{Sim}
 \providecommand{\terErrata}{Nao}
 \providecommand{\terListaFiguras}{Sim}
 \providecommand{\terListaQuadros}{Nao}
 \providecommand{\terListaTabelas}{Nao}
 \providecommand{\terSiglasAbrev}{Nao} 
 \providecommand{\terSimbolos}{Nao}
 \providecommand{\terResumos}{Sim}
 \providecommand{\terSumario}{Sim}
 \providecommand{\terAnexo}{Nao}
 \providecommand{\terApendice}{Nao}
 \providecommand{\terIndiceR}{Nao}
%----------------------------------------------------------------------------- 

\titulo{Premium Price: Aplicativo para Comparar Preços}
\autor{David Reksidler Júnior}
\local{Curitiba}
\data{2024} %Apenas ano 4 dígitos

% Orientador ou Orientadora
\orientador{Prof. Dr. Razer Anthom Nizer Rojas Montaño}
%Prof Emílio Eiji Kavamura, MSc}
%\orientadora{
%Prof\textordfeminine~Grace Kelly, DSc}
% Pode haver apenas uma orientadora ou um orientador
% Se houver os dois prevalece o feminino.

% Em termos de coorientação, podem haver até quatro neste modelo
% Sendo 2 mulhere e 2 homens.
% Coorientador ou Coorientadora
%\coorientador{}%Prof Morgan Freeman, DSc}
%\coorientadora{Prof\textordfeminine~Audrey Hepburn, DEng}

% Segundo Coorientador ou Segunda Coorientadora
\scoorientador{}
%Prof Jack Nicholson, DEng}
\scoorientadora{}
%Prof\textordfeminine~Ingrid Bergman, DEng}
% ----------------------------------------------------------
\addbibresource{referencias.bib}

% ----------------------------------------------------------
\instituicao{%
Universidade Federal do Paraná}

\def \ImprimirSetor{}%
%Setor de Tecnologia}

\def \ImprimirProgramaPos{}%Programa de Pós Graduação em Engenharia de Construção Civil}

\def \ImprimirCurso{}%
%Curso de Engenharia Civil}

\preambulo{
Trabalho de Conclusão de Curso apresentado ao curso de Pós-Graduação em Desenvolvimento Ágil de Software, Setor de Educação Profissional e Tecnológica, Universidade Federal do Paraná, como requisito parcial à obtenção do título de Especialista em Desenvolvimento Ágil de Software}

%Relatório Técnico apresentado ao curso de Especialização em Desenvolvimento Ágil de Software, Setor de Educação Profissional e Tecnológica, Universidade Federal do Paraná, como requisito parcial à obtenção do título de Especialista em Desenvolvimento Ágil de Software}

%Trabalho apresentado como requisito parcial para a obtenção do título de Mestre em Ciências pelo Programa de Pós Graduação em Engenharia de Construção Civil do  Setor de Tecnologia  da Universidade Federal do Paraná}
%do grau de Bacharel em Expressão Gráfica no curso de Expressão Gráfica, Setor de Exatas da Universidade Federal do Paraná}

%----------------------------------------------------------------------------- 

\newcommand{\imprimirCurso}{}
%Programa de P\'os Gradua\c{c}\~ao em Engenharia da Constru\c{c}\~ao Civil}

\newcommand{\imprimirDataDefesa}{
dd de MM de yyyy}

\newcommand{\imprimircdu}{
02:141:005.7}

% ----------------------------------------------------------
\newcommand{\imprimirerrata}{
Elemento opcional da \cites[4.2.1.2]{NBR14724:2011}. Exemplo:

\vspace{\onelineskip}

FERRIGNO, C. R. A. \textbf{Tratamento de neoplasias ósseas apendiculares com
reimplantação de enxerto ósseo autólogo autoclavado associado ao plasma
rico em plaquetas}: estudo crítico na cirurgia de preservação de membro em
cães. 2011. 128 f. Tese (Livre-Docência) - Faculdade de Medicina Veterinária e
Zootecnia, Universidade de São Paulo, São Paulo, 2011.

\begin{table}[htb]
\center
\footnotesize
\begin{tabular}{|p{1.4cm}|p{1cm}|p{3cm}|p{3cm}|}
  \hline
   \textbf{Folha} & \textbf{Linha}  & \textbf{Onde se lê}  & \textbf{Leia-se}  \\
    \hline
    1 & 10 & auto-conclavo & autoconclavo\\
   \hline
\end{tabular}
\end{table}}

% Comandos de dados - Data da apresentação
\providecommand{\imprimirdataapresentacaoRotulo}{}
\providecommand{\imprimirdataapresentacao}{}
\newcommand{\dataapresentacao}[2][\dataapresentacaoname]{\renewcommand{\dataapresentacao}{#2}}

% Comandos de dados - Nome do Curso
\providecommand{\imprimirnomedocursoRotulo}{}
\providecommand{\imprimirnomedocurso}{}
\newcommand{\nomedocurso}[2][\nomedocursoname]
  {\renewcommand{\imprimirnomedocursoRotulo}{#1}
\renewcommand{\imprimirnomedocurso}{#2}}


% ----------------------------------------------------------
\newcommand{\AssinaAprovacao}{

\assinatura{%\textbf
   {Professor(a)} \\ UFPR}
   %\assinatura{%\textbf
   %{Professora} \\ ENSEADE}
   %\assinatura{%\textbf
   %{Professora} \\ TIT}
   %\assinatura{%\textbf{Professor} \\ Convidado 4}
      
   \begin{center}
    \vspace*{0.5cm}
    %{\large\imprimirlocal}
    %\par
    %{\large\imprimirdata}
    \imprimirlocal, \imprimirDataDefesa.
    \vspace*{1cm}
  \end{center}
  }
  
% ----------------------------------------------------------
%\newcommand{\Errata}{%\color{blue}
%Elemento opcional da \textcite[4.2.1.2]{NBR14724:2011}. Exemplo:
%}

% ----------------------------------------------------------
\newcommand{\EpigrafeTexto}{%\color{blue}
\textit{``Tudo o que temos de decidir é o que fazer com o tempo que nos é dado.'
		(Gandalf, O Senhor dos Anéis - A Sociedade do Anel)}
}

% ----------------------------------------------------------
\newcommand{\ResumoTexto}{%\color{blue}
Com uma enorme variedade de produtos e serviços disponíveis no mercado, encontrar uma opção de qualidade e compreço justo acaba se tornando uma tarefa difícil e demorada. O presente trabalho tem como objetivo apresentar as informações relativas ao desenvolvimento de um aplicativo comparador de preços que permita aos usuários pesquisar e comparar os preços de diferentes produtos em várias lojas, facilitando e gerando mais confiança na tomada de decisão de compra e proporcionando economia de tempo e dinheiro aos consumidores.
} 

\newcommand{\PalavraschaveTexto}{%\color{blue}
preço; comparação; economia.}

% ----------------------------------------------------------
\newcommand{\AbstractTexto}{%\color{blue}
With a huge variety of products and services available on the market, finding a quality option at a fair price ends up becoming a difficult and time-consuming task. The present work has the objective to present information related to the development of a price comparison application that allows users to search and compare the prices of different products in several stores, facilitating and generating more confidence in making purchasing decisions and providing time and money savings to consumers.
}
% ---
\newcommand{\KeywordsTexto}{%\color{blue}
price. comparison. savings.
}

% ----------------------------------------------------------
\newcommand{\AgradecimentosTexto}{%\color{blue}
Durante a realização deste trabalho várias pessoas me ajudaram, sem as quais o mesmo não teria sido possível. A todas elas, as quais seria exaustivo aqui enunciar, a minha profunda
gratidão. A algumas delas, pelo apoio especial, gostaria de agradecer especialmente. Agradeço à minha esposa, Renata, pelo companheirismo, parceria, apoio e incentivo. 
Agradeço à toda a minha família, em especial meus pais, David e Patrícia, e meus irmãos, Kelvyn e Gabriel, por sempre acreditarem no meu potencial e incentivarem o meu desenvolvimento. Agradeço ao meu orientador, Razer Anthom Nizer Rojas, pelos ensinamentos, conselhos e pela paciência depositada em mim durante todo o período de orientação.
%Agradeço aos professores, Renato Carmo, André Luís Vignatti, André Luiz Pires Guedes e Leandro Miranda Zatesko pelas importantes contribuições e críticas que possibilitaram o desenvolvimento do meu trabalho.
}

% ----------------------------------------------------------
\newcommand{\DedicatoriaTexto}{%\color{blue}
\textit{ Este trabalho é dedicado às crianças adultas que,\\
   quando pequenas, sonharam em se tornar cientistas.}
	}

