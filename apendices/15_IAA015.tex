\label{ap:ap15}
\chapter{Tópicos em IA}
\section*{\textbf{A - ENUNCIADO}}

\subsection{Algoritmo Genético}


\textcolor{black}{Problema do Caixeiro Viajante}


\textcolor{black}{A Solução poderá ser apresentada em: Python (preferencialmente), ou em R, ou em Matlab, ou em C ou em
Java.}


\textcolor{black}{Considere o seguinte problema de otimização (a escolha do número de 100 cidades foi feita simplesmente
para tornar o problema intratável. A solução ótima para este problema não é conhecida).}


\textcolor{black}{Suponha que um caixeiro deva partir de sua cidade, visitar clientes em outras 99 cidades diferentes, e
então retornar à sua cidade. Dadas as coordenadas das 100 cidades, descubra o percurso de menor distância que passe uma
única vez por todas as cidades e retorne à cidade de origem.}


\textcolor{black}{Para tornar a coisa mais interessante, as coordenadas das cidades deverão ser sorteadas (aleatórias),
considere que cada cidade possui um par de coordenadas (x e y) em um espaço limitado de 100 por 100 pixels.}


\textcolor{black}{O relatório deverá conter no mínimo a primeira melhor solução (obtida aleatoriamente na geração da
população inicial) e a melhor solução obtida após um número mínimo de 1000 gerações. Gere as imagens em 2d dos pontos
(cidades) e do caminho.}


\textcolor{black}{Sugestão: }
\begin{enumerate}[series=listWWNumxxiii,label=(\arabic*),ref=\arabic*]
\item \textcolor{black}{considere o cromossomo formado pelas cidades, onde a cidade de início (escolhida aleatoriamente)
deverá estar na posição 0 e 100 e a ordem das cidades visitadas nas posições de 1 a 99 deverão ser definidas pelo
algoritmo genético.}
\item \textcolor{black}{A função de avaliação deverá minimizar a distância euclidiana entre as cidades (os pontos).}
\item \textcolor{black}{Utilize no mínimo uma população com 100 indivíduos;}
\item \textcolor{black}{Utilize no mínimo 1\% de novos indivíduos obtidos pelo operador de mutação;}
\item \textcolor{black}{Utilize no mínimo de 90\% de novos indivíduos obtidos pelo método de cruzamento (crossover-ox);}
\item \textcolor{black}{Preserve sempre a melhor solução de uma geração para outra.}
\end{enumerate}

\textbf{\textcolor{black}{Importante}}\textcolor{black}{: A solução deverá implementar os operadores de “cruzamento” e
“mutação”.}


\subsection{Compare a representação de dois modelos vetoriais}


\textcolor{black}{Pegue um texto relativamente pequeno, o objetivo será visualizar a representação vetorial, que poderá
ser um vetor por palavra ou por sentença. Seja qual for a situação, considere a quantidade de palavras ou sentenças
onde tenha no mínimo duas similares e no mínimo 6 textos, que deverão produzir no mínimo 6 vetores. Também limite o
número máximo, para que a visualização fique clara e objetiva.}


\textcolor{black}{O trabalho consiste em pegar os fragmentos de texto e codificá-las na forma vetorial. Após obter os
vetores, imprima-os em figuras (plot) que demonstrem a projeção desses vetores usando a PCA.}


\textcolor{black}{O PDF deverá conter o código-fonte e as imagens obtidas.}


%%%%%%%%%%%%%%%%%%%%%%%%%%%%%%%%%%%%%%%%%%%%%%%%%%%%%%%%%%%%%%%%%%%%%%%%%%%%%%%%%%%%%%%%%%%%%
\section*{\textbf{B - RESOLUÇÃO}}
\lipsum[30]