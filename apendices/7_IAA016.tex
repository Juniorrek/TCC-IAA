\label{ap:ap07}
\chapter{Aspectos filosóficos e éticos da IA}
\section*{\textbf{A - ENUNCIADO}}
%%%%%%%%%%%NO MODELO SO VEM DPS DE VISÃO COMPUTACIONAL

Título do Trabalho: {\textquotedbl}Estudo de Caso: Implicações Éticas do Uso do ChatGPT{\textquotedbl}



Trabalho em Grupo: O trabalho deverá ser realizado em grupo de alunos de no máximo seis (06) integrantes.



Objetivo do Trabalho: Investigar as implicações éticas do uso do ChatGPT em diferentes contextos e propor soluções
responsáveis para lidar com esses dilemas.

Parâmetros para elaboração do Trabalho:



\textbf{1. Relevância Ética}: O trabalho deve abordar questões éticas significativas relacionadas ao uso da inteligência
artificial, especialmente no contexto do ChatGPT. Os alunos devem identificar dilemas éticos relevantes e explorar como
esses dilemas afetam diferentes partes interessadas, como usuários, desenvolvedores e a sociedade em geral.

\textbf{2. Análise Crítica}: Os alunos devem realizar uma análise crítica das implicações éticas do uso do ChatGPT em
estudos de caso específicos. Eles devem examinar como o algoritmo pode influenciar a disseminação de informações, a
privacidade dos usuários e a tomada de decisões éticas. Além disso, devem considerar possíveis vieses algorítmicos,
discriminação e questões de responsabilidade.

\textbf{3. Soluções Responsáveis}: Além de identificar os desafios éticos, os alunos devem propor soluções responsáveis
e éticas para lidar com esses dilemas. Isso pode incluir sugestões para políticas, regulamentações ou práticas de
design que promovam o uso responsável da inteligência artificial. Eles devem considerar como essas soluções podem
equilibrar os interesses de diferentes partes interessadas e promover valores éticos fundamentais, como transparência,
justiça e privacidade.

\textbf{4. Colaboração e Discussão}: O trabalho deve envolver discussões em grupo e colaboração entre os alunos. Eles
devem compartilhar ideias, debater diferentes pontos de vista e chegar a conclusões informadas através do diálogo e da
reflexão mútua. O estudo de caso do ChatGPT pode servir como um ponto de partida para essas discussões, incentivando os
alunos a aplicar conceitos éticos e legais aprendidos ao analisar um caso concreto.

\textbf{5. Limite de Palavras}: O trabalho terá um limite de 6 a 10 páginas teria aproximadamente entre 1500 e 3000
palavras.

\textbf{6. Estruturação Adequada}: O trabalho siga uma estrutura adequada, incluindo introdução, desenvolvimento e
conclusão. Cada seção deve ocupar uma parte proporcional do total de páginas, com a introdução e a conclusão ocupando
menos espaço do que o desenvolvimento.

\textbf{7. Controle de Informações}: Evitar incluir informações desnecessárias que possam aumentar o comprimento do
trabalho sem contribuir significativamente para o conteúdo. Concentre-se em informações relevantes, argumentos sólidos
e evidências importantes para apoiar sua análise.

\textbf{8. Síntese e Clareza}: O trabalho deverá ser conciso e claro em sua escrita. Evite repetições desnecessárias e
redundâncias. Sintetize suas ideias e argumentos de forma eficaz para transmitir suas mensagens de maneira sucinta. 

\textbf{9. Formatação Adequada}: O trabalho deverá ser apresentado nas normas da ABNT de acordo com as diretrizes
fornecidas, incluindo margens, espaçamento, tamanho da fonte e estilo de citação. Deve-se seguir o seguinte template de
arquivo: hfps://bibliotecas.ufpr.br/wp- content/uploads/2022/03/template-artigo-de-periodico.docx


%%%%%%%%%%%%%%%%%%%%%%%%%%%%%%%%%%%%%%%%%%%%%%%%%%%%%%%%%%%%%%%%%%%%%%%%%%%%%%%%%%%%%%%%%%%%%
\section*{\textbf{B - RESOLUÇÃO}}
\lipsum[30]