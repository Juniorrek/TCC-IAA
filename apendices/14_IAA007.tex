\label{ap:ap14}
\chapter{Visualização de dados e storytelling}
\section*{\textbf{A - ENUNCIADO}}

\textcolor[HTML]{1D2125}{Escolha um conjunto de dados brutos (ou uma visualização de dados que você acredite que possa
ser melhorada) e faça uma visualização desses dados (de acordo com os dados escolhidos e com a ferramenta de sua
escolha)}

\textcolor[HTML]{1D2125}{Desenvolva uma narrativa/storytelling para essa visualização de dados considerando os conceitos
e informações que foram discutidas nesta disciplina. Não esqueça de deixar claro para seu possível público alvo
qual }\textbf{\textcolor[HTML]{1D2125}{o objetivo dessa visualização de dados, o que esses dados significam, quais
possíveis ações podem ser feitas com base neles.}}\textcolor[HTML]{1D2125}{ }

\textbf{\textcolor[HTML]{1D2125}{Entregue em um PDF:}}

\textcolor[HTML]{1D2125}{{}- O }\textbf{\textcolor[HTML]{1D2125}{conjunto de dados
brutos}}\textcolor[HTML]{1D2125}{ (}\textbf{\textcolor[HTML]{1D2125}{ou uma visualização de
dados }}\textcolor[HTML]{1D2125}{que você acredite que possa
ser }\textbf{\textcolor[HTML]{1D2125}{melhorada}}\textcolor[HTML]{1D2125}{);}

\textcolor[HTML]{1D2125}{{}- Explicação do }\textbf{\textcolor[HTML]{1D2125}{contexto e o
publico-alvo}}\textcolor[HTML]{1D2125}{ da visualização de dados e do storytelling que será desenvolvido;}

\textcolor[HTML]{1D2125}{{}- A }\textbf{\textcolor[HTML]{1D2125}{visualização desses
dados}}\textcolor[HTML]{1D2125}{ (de acordo com os dados escolhidos e com a ferramenta de sua
escolha) }\textbf{\textcolor[HTML]{1D2125}{explicando a escolha do tipo de visualização e da ferramenta
usada}}\textcolor[HTML]{1D2125}{; (}\textbf{\textcolor[HTML]{1D2125}{50 pontos}}\textcolor[HTML]{1D2125}{)}



%%%%%%%%%%%%%%%%%%%%%%%%%%%%%%%%%%%%%%%%%%%%%%%%%%%%%%%%%%%%%%%%%%%%%%%%%%%%%%%%%%%%%%%%%%%%%
\section*{\textbf{B - RESOLUÇÃO}}
\lipsum[30]