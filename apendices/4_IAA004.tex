\label{ap:ap04}
\chapter{Estatística Aplicada I}
\section*{\textbf{A - ENUNCIADO}}
\begin{enumerate}[series=listWWNumxxvi,label=\arabic*),ref=\arabic*]
\item \textbf{Gráficos e tabelas}
\end{enumerate}

\bigskip

(15 pontos) Elaborar os gráficos box-plot e histograma das variáveis “age” (idade da esposa) e “husage” (idade do
marido) e comparar os resultados

(15 pontos) Elaborar a tabela de frequencias das variáveis “age” (idade da esposa) e “husage” (idade do marido) e
comparar os resultados


\bigskip

\begin{enumerate}[resume*=listWWNumxxvi]
\item \textbf{Medidas de posição e dispersão}
\end{enumerate}

\bigskip

(15 pontos) Calcular a média, mediana e moda das variáveis “age” (idade da esposa) e “husage” (idade do marido) e
comparar os resultados

(15 pontos) Calcular a variância, \ desvio padrão e coeficiente de variação das variáveis “age” (idade da esposa) e
“husage” (idade do marido) e comparar os resultados


\bigskip

\begin{enumerate}[resume*=listWWNumxxvi]
\item \textbf{Testes paramétricos ou não paramétricos}
\end{enumerate}

\bigskip

(40 pontos) Testar se as médias (se você escolher o teste paramétrico) \ ou as medianas (se você escolher o teste não
paramétrico) das variáveis “age” (idade da esposa) e “husage” (idade do marido) são iguais, construir os intervalos de
confiança e comparar os resultados.

Obs: 

Você deve fazer os testes necessários (e mostra-los no documento pdf) para saber se você deve usar o unpaired test
(paramétrico) ou o teste U de Mann-Whitney (não paramétrico), justifique sua resposta sobre a escolha.

Lembre-se de que os intervalos de confiança já são mostrados nos resultados dos testes citados no item 1 acima. 


%%%%%%%%%%%%%%%%%%%%%%%%%%%%%%%%%%%%%%%%%%%%%%%%%%%%%%%%%%%%%%%%%%%%%%%%%%%%%%%%%%%%%%%%%%%%%
\section*{\textbf{B - RESOLUÇÃO}}
\lipsum[30]